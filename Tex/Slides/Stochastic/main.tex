%!TEX program = xelatex
\documentclass[10pt, compress]{beamer}
\usetheme[titleprogressbar]{m}
\usepackage{booktabs}
\usepackage[italian]{babel}
\usepackage[latin1]{inputenc}
\usepackage[scale=2]{ccicons}
\usepackage[utf8]{inputenc}
% \usepackage{enumitem}
\usepackage{multicol}
\usepackage{pst-node}
\usepackage[inline]{enumitem}
% \usepackage{tikz}
\usepackage{biblatex} %Imports biblatex package
\addbibresource{sample.bib} %Import the bibliography file
\usepackage{tikz}
\usetikzlibrary{matrix}
\usepackage{tikz-cd}
\usepackage{pgfplots}
\usepackage{pst-node}
%\uspackage{auto-pst-pdf}
\usepackage{tikz-cd} 
\usepackage{graphicx, animate}
\usepackage{graphicx}
\pgfplotsset{%
    ,compat=1.12
    ,every axis x label/.style={at={(current axis.right of origin)},anchor=north west}
    ,every axis y label/.style={at={(current axis.above origin)},anchor=north east}
    }
\newlist{inlineroman}{enumerate*}{1}
\setlist[inlineroman]{itemjoin*={{, and }},afterlabel=~,label=\Roman*.}

\newcommand{\inlinerom}[1]{
\begin{inlineroman}
#1
\end{inlineroman}
}

\renewcommand{\thefootnote}{[\arabic{footnote}]}% Modify footnote globally

%%%simboli speciali
\newcommand{\N}{\mathbb{N}}
\newcommand{\Z}{\mathbb{Z}}
\newcommand{\Q}{\mathbb{Q}}
\newcommand{\R}{\mathbb{R}}
\newcommand{\C}{\mathbb{C}}

\newcommand{\CO}{\mathcal{C}}

\DeclareMathOperator{\Dom}{Dom}
\DeclareMathOperator{\Ima}{Im}

\DeclareMathOperator{\mcd}{MCD}
\DeclareMathOperator{\mcm}{mcm}
\DeclareMathOperator{\supp}{supp}
\DeclareMathOperator{\spanlin}{span}

%\DeclarePairedDelimiter\abs{\lvert}{\rvert}
%\DeclarePairedDelimiter\norm{\lVert}{\rVert}

%Necessario per far si che la dimensione del valore assoluto e della norma si adatti all'aromento passato
%\makeatletter
%\let\oldabs\abs
%\def\abs{\@ifstar{\oldabs}{\oldabs*}}
%
%\let\oldnorm\norm
%\def\norm{\@ifstar{\oldnorm}{\oldnorm*}}
%\makeatother

%rinomino il comando per il prodotto interno
\let\pint\braket
%comando per il prodotto scalare
\newcommand{\pscal}[2]{\left\langle #1, #2 \right\rangle}

%Spaziatura per i quantificatori
\let\oldforall\forall
\renewcommand{\forall}{\; \oldforall \;}
\let\oldexists\exists
\renewcommand{\exists}{\; \oldexists \;}

%freccia per la convergenza debole
\newcommand{\longrightharpoonup}{\xrightharpoonup{\phantom{AB}}}

\newcommand{\warrow}{\xrightharpoonup{\:\, w \:\, }}
\newcommand{\sarrow}{\overset{s}{\longrightarrow}}
\newcommand{\wastarrow}{\xrightharpoonup{ w^* }}

\newcommand{\ctikz}[1]{
\begin{center}
\begin{tikzpicture}
\node{#1}
\end{tikzpicture}
\end{center}
}

\newcommand{\ctikzo}[1]{
\begin{center}
\begin{tikzpicture}
\node[opacity=0]{#1}
\end{tikzpicture}
\end{center}
}

\newcommand{\tikzz}[1]{
\begin{tikzpicture}
\node{#1}
\end{tikzpicture}
}

\newcommand{\tikzo}[1]{
\begin{tikzpicture}
\node[opacity=0]{#1}
\end{tikzpicture}
}

\newcommand{\ctikzi}[1]{
\begin{center}
\begin{tikzpicture}
\node[opacity=0.2]{#1}
\end{tikzpicture}
\end{center}
}

\newcommand{\tikzi}[1]{
\begin{tikzpicture}
\node[opacity=0.2]{#1}
\end{tikzpicture}
}

\newcommand{\semitransp}[2][1]{\textcolor{#1}{#2}}

\newcommand{\comment}[1]{}
\usepgfplotslibrary{dateplot}

\title{Optimal control via dynamic programming}
\subtitle{Stochastic problem}
\author{Andrea Scalenghe}
\institute{Tesi magistrale}
%\date{22 Luglio 2022}
\begin{document}

\maketitle

\begin{frame}{Stochastic Optimal Control}
    What if the system evolves and is controlled stochastically? We enter the reign of stochastic optimal control. 

    We will study controlled diffusion processes, which can be represented by:

    \begin{equation}\label{SDE}
        dx(r) = f(r,x(r),u(r))dr + \sigma(r,x(r),u(r))dw(r),\, r\in I_0
    \end{equation}

    where $I_0$ is a time interval and $f,\sigma$ are drift and volatility coefficients. The control $u$ is itself a random process.
\end{frame}

\begin{frame}{Markov Processes}
    \begin{definition}\label{2-1-markovprocessdef}
    A stochastic process $x$ satisfies the Markov property if there exists a function 
    $p:I_0\times \Sigma\times I_0\times  \mathcal{B}(\Sigma)\rightarrow \R$ such that:


        - For all $t,s,B$ the function $x\mapsto p(t,x,s,B)$ is Borel measurable on $\Sigma$ and for all $t,x,s$ the function $A\mapsto p(t,x,s,B)$ is a probability measure on $(\Omega,\mathcal{F})$. The Chapman-Kolmogorov equation holds for all $s,t,r\in I_0$ such that $t<r<s$:
        \begin{equation}\label{2-1-markovprocessdef-chapkol}
            p(t,x,s,B) = \int_{\Sigma} p(r,y,s,B)\,p(t,x,r,dy)
        \end{equation}


    And such that for all $r,s\in I_0$ where $r,s$ and for all $B\in\mathcal{B}(\Sigma)$ then:

    \begin{equation}\label{2-1-markovprocessdef-condonp}
        P(x(s)\in B\,|\,\mathcal{F}_r^x) = p(r,x(r),s,B)
    \end{equation}

    Where $\mathcal{F}_r^x=\sigma\left(x(l)\,:\,l\in[t_0,r]\right)$.
\end{definition}

% Function $p$ is called \textit{Markov Transition Kernel}
\end{frame}

% \begin{frame}{Markov Property}
%     A process $x$ exhibits the Markov property if:
%     \begin{itemize}
%         \item $x \mapsto p(t,x,s,B)$ is Borel measurable.
%         \item $A \mapsto p(t,x,s,B)$ is a probability measure.
%         \item Chapman-Kolmogorov equation is satisfied for all $s,t,r \in I_0$ with $t<r<s$.
%     \end{itemize}
%     This function $p$ is known as the Markov Transition Kernel.
% \end{frame}

\begin{frame}{Markov Transition Kernel}
    Heuristically, the Markov probability kernel gives the probability that the system starting from $x$ at time $t$ will be in $B$ at time $s$. We define the expected value of a function of the process given the initial data $(t,x)$ as:
    \[E_{tx}\phi(x(s)) = \int_{\Sigma} \phi(y)\,p(t,x,s,dy)\]
    for a real-valued Borel-measurable function $\phi$.

    This gives rise to a linear operator over (somehow integrable) functions:

    \[T_{t,s}\phi(x) = E_{tx}\phi(x(s))\]
\end{frame}

\begin{frame}{Diffusion Processes}
    Generally, a diffusion process $x$ is a Markov process with continuous paths. We also impose the existence of the following functions $a_{ij}(t,x),f_{ij}(t,x)$ and limits:

    \begin{equation}
            \lim_{h\to0^+} \int_{\abs{x-y}>\epsilon} \,p(t,x,t+h,dy) = 0
    \end{equation}

        \begin{equation}
            \lim_{h\to0^+} \int_{\abs{x-y}\leq\epsilon} (y_i-x_i)\,p(t,x,t+h,dy) = f_i(t,x)
        \end{equation}

        \begin{equation}
            \lim_{h\to0^+} \int_{\abs{x-y}\leq\epsilon} (y_i-x_i)(y_j-x_j)\,p(t,x,t+h,dy) = a_{ij}(t,x).
        \end{equation}

        Under stricter conditions on $f,a$ a diffusion process is described by \ref{SDE}.
    
    % It is completely characterized by its local drift $f$ and local covariance $a$ functions. It is described by the stochastic differential equation:
    % \[dx(s) = f(s,x(s))ds + \sqrt{a}(s,x(s))dw(s)\]
\end{frame}

% \begin{frame}{Linear Operators in Markov Processes}
%     The linear operator $T_{t,s}$ associated with a Markov process is given by:
%     \[T_{t,s}\phi(x) = E_{tx}\phi(x(s))\]
%     It represents the expected value of $\phi$ at time $s$, starting from $x$ at time $t$.
% \end{frame}

\begin{frame}{Backward Evolution Operator}
    The backward evolution operator $A$ is defined for measurable functions $\Phi$ as:
    \begin{equation}A\Phi(t,x) = \lim_{h\to0+} \frac{E_{tx}\Phi(t+h,x(t+h))-\Phi(t,x)}{h}\end{equation}
    % It plays a crucial role in connecting the dynamics of Markov processes with differential equations.
    We denote $\mathcal{D}(A)$ for the space of function such that the limit exists. In various contexts, Dynkin's formula holds:

    \begin{equation}\label{2-1-dynkform}
            E_{tx}\Phi(s,x(s)) - \Phi(t,x) = E_{tx}\int_t^s A\Phi(r,x(r)) \,dr
        \end{equation}
\end{frame}

\begin{frame}{Dynkin's Formula for diffusion processes}
    We get Dynkin by taking $E_{tx}$ over:
    \begin{align}\label{dynk-deriv}
    \Phi(s,x(s)) & = \Phi(t,x) + \int_t^s\Phi_s(r,x(r))\,dr \\
    % & + \sum_{i=1}^n \left[\int_t^s \Phi_{x^i}(r,x(r))f_i(r,x(r))\,dr + \sum_{j=1}^n\int_t^s\Phi_{x^i}(r,x(r))\sigma_{ij}(r,x(r))\,dw^j(r)\right] \\
    % & + \frac{1}{2}\sum_{i,j=1}^n \sum_{l=1}^n\int_t^s\Phi_{x^ix^j}(r,x(r))\sigma_{il}(r,x(r))\sigma_{jl}(r,x(r))\,dr \\
    % & = \Phi(t,x) + \int_t^s \Phi_s(r,x(r)) + D_x\Phi\cdot f(r,x(r)) \\
    % & +\int_t^s \frac{1}{2}D^2_x\Phi\cdot a(r,x(r)) \,dr +  D_x\Phi\cdot\sigma(r,x(r)) \,dw(r) \\
    & = \Phi(t,x) + \int_t^s A\Phi(r,x(r))\,dr \\
    & +\int_t^s D_x\Phi\cdot\sigma(r,x(r)) \,dw(r),
    % & = \Phi(t,x) + \int_t^s A\Phi(r,x(r))\,dr + \int_t^s D_x\Phi\cdot\sigma(r,x(r)) \,dw(r)
\end{align}

where the last (stochastic) integral can be seen as a martingale. In particular, if we take $\Phi$ to have polynomial growth then $D_x\Phi\cdot\sigma\in\mathbb{L}^2(I_0)$\footnote{using estimates on SDE solutions.}.


\end{frame}

\begin{frame}{Markov Control Processes}
    Moving to controlled dynamics, the controlled SDE becomes:
    \[dx(r) = f(r,x(r),u(r))dr + \sigma(r,x(r),u(r))dw(r),\, r\in I_0\]
    
    The objective is to minimize a cost criterion involving a running cost $L$ and terminal cost $\Psi$.
    
    \[J(t,x;u) = E_{tx}\left\{\int_t^{\tau} L(s,x(s),u(s))\,ds + \Psi(\tau,x(\tau))\right\}\]

    The operator $A$ is defined for $\Phi\in C^{1,2}_p(\overline{Q}_0)$ as:

    \begin{equation}\label{2-1-defAwithv}
    A^{v}\Phi = \Phi_t + \sum_{i=1}^nf_i(t,x,v)\Phi_{x_i} + \frac{1}{2}\sum_{i,j=1}^n a_{ij}(t,x,v)\Phi_{x_ix_j}
    \end{equation}
\end{frame}

\begin{frame}{Optimality}
    \begin{definition}
    A reference probability system is a tuple $(\Omega,\{\mathcal{F}_s\},P,\omega)$ such that $\nu=(\Omega,\mathcal{F}_{t_1},P)$ is a probability space, $\{\mathcal{F}_s\}$ is a filtration on $\Omega$, $w$ is an $\mathcal{F}$-adapted Brownian motion on $[t,t_1]$.

    We denote with $\mathcal{A}_{t\nu}$ progressively measurable admissible controls $u$. We define:

    \begin{equation}\label{2-1-systemopt}
        V_{\nu} = \inf_{u\in\mathcal{A}_{t\nu}} J(t,x;u)
    \end{equation}

    \begin{equation}\label{2-1-oprt}
        V_{PM} = \inf_{\nu} V_{\nu}.
    \end{equation}

    Equation \ref{2-1-systemopt} and \label{2-1-oprt} respectively define $\nu$-\textit{optimality} and \textit{optimality}    \end{definition}
\end{frame}

% \begin{frame}{Objective and Hamiltonian}
%     The cost to minimize is:
    
%     The Hamiltonian for this optimization problem integrates the running cost, terminal cost, and the dynamics of the process.
% \end{frame}

\begin{frame}{HJB equation derivation}
    Let us suppose that $O=\R^n$, then by the dynamic programming principle for every $h<t_1-t$:
    \[V(t,x) = \inf_{u\in\mathcal{A}}E_{tx}\left\{\int_t^{t+h} L(s,x(s),u(s))\,ds + V(t+h,x(t+h))\right\}.\]
    If we take the constant control $u\equiv v$ then by Dynkin's formula we get:
    \[
        0 \leq E_{tx}\int_t^{t+h} A^vV(s,x(s))\,ds + E_{tx}\int_t^{t+h} L(s,x(s),v)\,ds,
    \]
    dividing by $h$ and taking the limit for $h\to0^+$:
    \[0 \leq A^vV(t,x) + L(t,x,v).\]
\end{frame}

\begin{frame}{Hamiltonian and HJB equation}
    The Hamiltonian $\mathcal{H}$ is defined as:
    \[\mathcal{H}(t,x,p,A) = \sup_{v\in U}\left\{-f\cdot p - \frac{1}{2}\text{tr}\left[a\cdot A\right] - L\right\}\]
    The verification theorem uses this Hamiltonian to provide conditions under which a control is optimal. We'll impose the optimality condition:

    \[A^vW = 0 \Rightarrow -\frac{\partial W}{\partial t} + \mathcal{H}(t,x,D_xW,D_x^2W) = 0\]
\end{frame}

\begin{frame}{Verification Theorem}
    \begin{theorem}
    Let $W\in C^{1,2}(Q)\cap C_p(\overline{Q})$ such that:
    \begin{equation}\label{systverstoch}\begin{cases}
        -\frac{\partial W}{\partial t} + \mathcal{H}(t,x,D_xW,D_x^2W) = 0, & \forall(t,x)\in Q \\
        V(t,x) = \Psi(t,x), & \forall(t,x)\in\partial Q.
    \end{cases}\end{equation}
    Then, for any system $\nu$, initial condition $(t,x)\in Q$ and any $u\in \mathcal{A}_{t\nu}$ we have $W(t,x)\leq J(t,x;u)$. If there exists a system $\nu^{\ast}$ and a control $u^{\ast}$ which realizes the Hamiltonian's minimum, then:
        \begin{equation}
            V_{PM}(t,x) = J(t,x;u^{\ast}).
        \end{equation}
    \end{theorem}
\end{frame}
% \begin{frame}{Conclusion}
%     Markov diffusion and control processes provide a rich framework for modeling and optimizing stochastic systems. Through the lens of Markov properties, transition kernels, and associated operators, we gain powerful tools for analysis and control of such systems.
% \end{frame}

\begin{frame}{Proof Idea}
    For $O$ bounded and $W\in C^{1,2}(\overline{Q})$ we can apply Ito and get the thesis as we did in Dynkin \ref{dynk-deriv}.

    In $O$ is unbounded we define:

    \[O_{\rho} = O\cap\left\{\abs{x}<\rho\,|\,d(x,\partial O)>\frac{1}{\rho}\right\},\, Q_{\rho} = [t_0,t_1-\rho^{-1}]\times O_{\rho},\]

    get thesis for all $\rho^{-1}<t_1-t_0$ and pass to the limit showing uniform integrability of both addenda, which together with convergence in probability implies $L^1$ convergence.
\end{frame}

\end{document}