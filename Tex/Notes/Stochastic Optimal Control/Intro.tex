\documentclass[12pt,a4paper]{book}
% \usepackage{times}
\usepackage[a4paper, margin=0.6in]{geometry}
\usepackage[utf8]{inputenc}
\usepackage[english]{babel}
\selectlanguage{english}
\usepackage{amsmath,amsthm,amssymb,fge,mathrsfs}
\usepackage{mathtools}
\usepackage{hyperref}
\usepackage{makeidx} %per fare l'indice
\usepackage{faktor} %per i quozienti

%tikz (pacchetto per i disegni)
\usepackage{tikz}
\usetikzlibrary{arrows}
\usetikzlibrary{tikzmark}
\usetikzlibrary{calc} %per poter fare i calcoli
\usetikzlibrary{arrows.meta} %per usare veri tipi di frecce
\usetikzlibrary{calc,patterns,angles,quotes} %per disegnare gli angoli
\usetikzlibrary{decorations} %per i grafici orientati (con sopra le frecce)
\usetikzlibrary{decorations.markings}
\usetikzlibrary{backgrounds} %per il colore sullo sfondo delle immagini
\usetikzlibrary{shapes.geometric} %per i triangoli negli alberi

\usepackage{pgfplots}
\pgfplotsset{compat=newest}
\usepgfplotslibrary{fillbetween} %per colorare le aree comprese tra due grafici

\usepackage{caption} %per la caption sotto alle immagini
\usepackage[tracking=true]{microtype} %per avvicinare il testo
\usepackage{enumitem} %per modificare agilmente gli elenchi
\usepackage{esvect} %per le freccette di vettore sui caratteri
\usepackage{stmaryrd} %fulmine per il simbolo dell'assurdo
\usepackage{relsize} %per fare i simboli più grandi
\usepackage{mathabx} %per il simbolo di dotminus
\usepackage{esint} %per l'integrale tagliato (fint)

\usepackage[customcolors]{hf-tikz} %per evidenziare pezzi di matrice

\usepackage{verbatim} %per commentare un blocco

\usepackage{multirow} %per la tabella
\usepackage{array} %per lo spazio nella tabella

%Libreria per gli alberi
\usepackage[linguistics]{forest}

%Librerie aggiunte
\usepackage{multicol} %per gli elenchi puntati su più colonne/righe
\usepackage{multirow}

%pacchetto per il prodotto interno
\usepackage{physics}

%Libreria per i subfiles

\usepackage{subfiles} %meglio caricarla per ultima

\usepackage{biblatex}
\addbibresource{biblio.bib}
%usiamo paragraph come nuova sottosottosottosezione 
\makeatletter
\renewcommand\paragraph{\@startsection{paragraph}{4}{\z@}%
            {-2.5ex\@plus -1ex \@minus -.25ex}%
            {1.25ex \@plus .25ex}%
            {\normalfont\normalsize\bfseries}}
\makeatother
\setcounter{secnumdepth}{4} % how many sectioning levels to assign numbers to
\setcounter{tocdepth}{4}    % how many sectioning levels to show in ToC

\newtheorem{theorem}{Theorem}
\numberwithin{theorem}{section}
\newtheorem*{theorem*}{Theorem}
\newtheorem{corollary}[theorem]{Corollary}
\newtheorem{proposition}[theorem]{Proposition}
\newtheorem{lemma}[theorem]{Lemma}
\newtheorem*{lemma*}{Lemma}

\newtheorem{definition}{Definition}
\numberwithin{definition}{section}

\newtheorem*{remark}{Remark}

\newtheorem{example}{Example}
\numberwithin{example}{section}

\newtheorem{exercise}{Exercise}
\numberwithin{exercise}{section}

\renewcommand{\proofname}{\proof}

\renewcommand{\thefootnote}{[\arabic{footnote}]}% Modify footnote globally

%%%simboli speciali
\newcommand{\N}{\mathbb{N}}
\newcommand{\Z}{\mathbb{Z}}
\newcommand{\Q}{\mathbb{Q}}
\newcommand{\R}{\mathbb{R}}
\newcommand{\C}{\mathbb{C}}

\newcommand{\CO}{\mathcal{C}}

\DeclareMathOperator{\Dom}{Dom}
\DeclareMathOperator{\Ima}{Im}

\DeclareMathOperator{\mcd}{MCD}
\DeclareMathOperator{\mcm}{mcm}
\DeclareMathOperator{\supp}{supp}
\DeclareMathOperator{\spanlin}{span}

%\DeclarePairedDelimiter\abs{\lvert}{\rvert}
%\DeclarePairedDelimiter\norm{\lVert}{\rVert}

%Necessario per far si che la dimensione del valore assoluto e della norma si adatti all'aromento passato
%\makeatletter
%\let\oldabs\abs
%\def\abs{\@ifstar{\oldabs}{\oldabs*}}
%
%\let\oldnorm\norm
%\def\norm{\@ifstar{\oldnorm}{\oldnorm*}}
%\makeatother

%rinomino il comando per il prodotto interno
\let\pint\braket
%comando per il prodotto scalare
\newcommand{\pscal}[2]{\left\langle #1, #2 \right\rangle}

%Spaziatura per i quantificatori
\let\oldforall\forall
\renewcommand{\forall}{\; \oldforall \;}
\let\oldexists\exists
\renewcommand{\exists}{\; \oldexists \;}

%freccia per la convergenza debole
\newcommand{\longrightharpoonup}{\xrightharpoonup{\phantom{AB}}}

\newcommand{\warrow}{\xrightharpoonup{\:\, w \:\, }}
\newcommand{\sarrow}{\overset{s}{\longrightarrow}}
\newcommand{\wastarrow}{\xrightharpoonup{ w^* }}

\DeclareMathOperator*{\argmax}{arg\,max}
\DeclareMathOperator*{\argmin}{arg\,min}

\setlength{\parindent}{0in} 
\title{Title}
\author{Andrea Scalenghe}
\date{March 2024}



\begin{document}

%I want to understand how to proceed also in view of stochastics

\section{Introduction}

We now study stochastic optimal control. The system we aim to control is governed by stochastic differential equations. 

Short description of what is done in this chapter.

\section{Markov diffusion process}

I now recall some definitions, give new ones and set the notation. Let $\Sigma\subseteq\R^n$ and $\mathcal{B}(\Sigma)$ 
the associated Borel $\sigma$-algebra. Let $(\Omega, \mathcal{F}, P)$ a general probability space. 
Given $x(s,\omega)$ a $\Sigma$-valued random process from $I_0=[t_0,t_1)$ and $(\Omega,\mathcal{F})$, let us denote by:

\[P(C\,|\,x(s_1),\dots,x(s_m)),\,C\in\mathcal{F}\]

The conditional probability of $C$ given the sigma algebra $\bigvee_{i=1}^m\sigma(x(s_i))$.

\begin{definition}\label{2-1-markovprocessdef}
    A stochastic process $x$ satisfies the Markov property if there exists a function 
    $p:I_0\times \Sigma\times I_0\times  \mathcal{B}(\Sigma)\rightarrow \R$ such that:

    \begin{enumerate}
        \item For all $t,s,B$ the function $x\mapsto p(t,x,s,B)$ is borel measurable on $\Sigma$
        \item For all $t,x,s$ the function $A\mapsto p(t,x,s,B)$ is a probability measure on $(\Omega,\mathcal{F})$
        \item The Chapman-Kolmogorov equation holds for all $s,t,r\in I_0$ such that $t<r<s$:
        \begin{equation}\label{2-1-markovprocessdef-chapkol}
            p(t,x,s,B) = \int_{\Sigma} p(r,y,s,B)\,p(t,x,r,dy)
        \end{equation}
    \end{enumerate}

    And such that for all $r,s\in I_0$ where $r,s$ and for all $B\in\mathcal{B}(\Sigma)$ then:

    \begin{equation}\label{2-1-markovprocessdef-condonp}
        P(x(s)\in B\,|\,\mathcal{F}_r^x) = p(r,x(r),s,B)
    \end{equation}

    Where $\mathcal{F}_r^x=\sigma\left(x(l)\,:\,l\in[t_0,r]\right)$.
\end{definition}

Function $p$ is called \textit{Markov Transition Kernel}. We shall see a Markov transition kernel as 
the probability that the system starting from $x$ at time $t$ will be in $B$ at time $s$. This heuristic interpretation 
clarifies the following notation:

\begin{equation}
    E_{tx}\phi(x(s)) = \int_{\Sigma} \phi(y)\,p(t,x,s,dy) 
    % p(t,x,s,B) = P(x(s)\in B\,|\,)
\end{equation}

For a real valued borel-measurable function $\phi$. Given a Markov process $x$ we can define a family of 
linear operators associated to it. Let $t<s$, hereafter all time indices will always be in $I_0$, and define:

\begin{equation}
    T_{t,s}\phi(x) = \int_{\Sigma} \phi(y)\,p(t,x,s,dy) = E_{tx}\phi(x(s))
\end{equation}

Integrability assumptions on $\phi$ vary from case to case. For now, we can take $\phi$ to be bounded. 
Because of Chapman-Kolmogorov equation \ref{2-1-markovprocessdef-chapkol} the family $(T_{t,s})_{t,s\in I_0}$ 
satisfies the property:

\begin{equation}\label{2-1-propT}
    T_{tr}\left[T_{rs}\phi\right] = T_{ts}\phi
\end{equation}

For all $t<r<s$. This family of linear operators defines another operator, the \textit{backward evolution operator}.
Let $A:\left\{\Phi:I_0\times\Sigma\rightarrow\R\right\}\rightarrow \R$:

\begin{equation}\label{2-1-backwarddef}
    A\Phi(t,x) = \lim_{h\to0+} \frac{E_{tx}\Phi(t+h,x(t+h))-\Phi(t,x)}{h}
\end{equation}

provided that the limit exists. We define $\mathcal{D}(A)$ the space of functions such that limit \ref{2-1-backwarddef} exists. 
The following holds.

\begin{proposition}
    Let $A$ as before, then for all $\Phi\in\mathcal{A}$ the following hold:

    \begin{enumerate}
        \item $\Phi,\frac{\partial\Phi}{\partial t}$ and $A\Phi$ are continuous
        \item For all $t,s\in\overline{I}_0$, $t<s$ then:
        
        \[E_{tx}\abs{\Phi(s,x(s))}< +\infty,E_{tx}\int_t^s\abs{A\Phi(r,x(r))}\,dr < +\infty\]
        
        \item Dynkin's formula holds for all $t<s$:
        
        \begin{equation}\label{2-1-dynkform}
            E_{tx}\Phi(s,x(s)) - \Phi(t,x) = E_{tx}\int_t^s A\Phi(r,x(r)) \,dr
        \end{equation}
    \end{enumerate}

    % \begin{proof}
    %     I prove Dynkin's formula in the case of $T_{ts}$ being a Feller semigroup. 
    %     %Dynkin, E. B., Markov processes,  

    % \end{proof}
\end{proposition}

Dynkin's formula can be proved in different instances, subject to the nature of the random process. We will see 
that it is a natural consequence of Ito formula for continuous state space processes.

If the random process $x$ is autonomous (time-homogeneous) then the linear operator family is a 
semigroup. Recall that a Markov process is homogeneous if for all $t<s$ in $I_0$ then:

\[p(t,x,s,B) = p(0,x,s-t,B)\]

If so, by calling $T_s=T_{0s}$ property \ref{2-1-propT} is:

\begin{align}
    T_{s+r}\phi(x) & = \int_{\Sigma} \phi(y)\,p(0,x,s+r,dy) \\
    % & = \int_{\Sigma} \phi(y)\,p(r,x,s,dy) \\
    & = \int_{\Sigma}\phi(y)\int_{\Sigma} p(r,z,r+s,dy)\,p(0,x,r,dz) \\
    & = \int_{\Sigma}\int_{\Sigma} \phi(y) p(r,z,r+s,dy)\,p(0,x,r,dz) \\
    & = \int_{\Sigma}\int_{\Sigma} \phi(y) p(0,z,s,dy)\,p(0,x,r,dz) \\
    & = \int_{\Sigma} T_{s}\phi(z) \,p(0,x,r,dz) \\
    & = T_r\left[T_s\phi(x)\right].
\end{align}

While the backward evolution operator analogous is called the \textit{generator} and is defined as:

\begin{equation}
    G\phi(x) = - \lim_{h\to0^+} \frac{T_h\phi(x) - \phi(x)}{h}
\end{equation}

With $D(G)$ as $\mathcal{D}(A)$ before. It is worth noting that, formally, the following equality holds:

\begin{equation}
    A\Phi = \frac{\partial \Phi}{\partial t} - G\Phi(t,\cdot)
\end{equation}

This relation links the two operators and the autonomous to the non-autonomous case. We now turn our attention to 
a subset of Markov processes: diffusion processes. A diffusion process is a Markov process whose paths are continuous. 
More formally. 

\begin{definition}
    A diffusion process $x:\overline{I}_0\times\Omega\rightarrow\Sigma$ is a almost surely continuous Markov process with Markov transition kernel $p$ such that:

    \begin{itemize}
        \item For every $\epsilon>0$:
        
        \begin{equation}
            \lim_{h\to0^+} \int_{\abs{x-y}>\epsilon} \,p(t,x,t+h,dy) = 0
        \end{equation}

        \item There exist functions $a_{ij}(t,x),f_{ij}(t,x)$ for $(t,x)\in\overline{Q}_0$ and $i,j=1,\dots,n$ such that for every $\epsilon>0$:
        
        \begin{equation}
            \lim_{h\to0^+} \int_{\abs{x-y}\leq\epsilon} (y_i-x_i)\,p(t,x,t+h,dy) = f_i(t,x)
        \end{equation}

        And:

        \begin{equation}
            \lim_{h\to0^+} \int_{\abs{x-y}\leq\epsilon} (y_i-x_i)(y_j-x_j)\,p(t,x,t+h,dy) = a_{ij}(t,x).
        \end{equation}
    \end{itemize}

    These limits are intended uniformly.
\end{definition}

Functions $f=(f_1,\dots,f_n)$ and $a=(a_{ij})_{ij}$ are respectively called local drift and local covariance coefficients.

How does the backward evolution operator, and the generator in the autonomous case, adapt to this situation? 
To answer this question we reduce our problem to a stochastic differential one by relying on the differential structure of a diffusion process. 
Give the local drift and covariance $f,a$ of a diffusion process $x$ we claim that it satisfies:

\begin{equation}\label{2-1-SDEdef}
    dx(s) = f(s,x(s))ds + \sqrt{a}(s,x(s))dw(s) 
\end{equation}

Clearly we have to impose further conditions of the stochastic differential equation's coefficients to ensure existence of a solution. 
In particular, we want those coefficients to be Lipshitz and sub-linearly growing with respect to the second variable. In equation \ref{2-1-SDEdef} 
We define the square root of $a$ as a function $\sqrt{a}=\sigma$ such that:

\begin{equation}
    \sigma(t,x)\cdot\sigma'(t,x) = a(t,x)
\end{equation}

We recall that under existence hypothesis for every $\Phi\in C^{1,2}(\overline{Q}_0)$ Ito's formula holds:

\begin{equation}\label{2-1-itotox}
    d\Phi(s,x(s)) = \Phi_s(s,x(s))ds + \sum_{i=1}^n \Phi_{x^i}(s,x(s)) dx^i(s) + \frac{1}{2}\sum_{i,j=1}^n \Phi_{x^ix^j}(s,x(s)) d[x,x]^{ij}(s)
\end{equation}

where $[x,y]$ is the covariation of process $x$ and $y$. Recall that this relation has always to be intended in integral form, that is:

\begin{equation}
    \Phi(s,x(s)) = \Phi(t,x) + \int_t^s\Phi_s(r,x(r))\,dr + \sum_{i=1}^n \int_t^s \Phi_{x^i}(r,x(r))\,dx^i(r) + \frac{1}{2}\sum_{i,j=1}^n \int_t^s\Phi_{x^ix^j}(r,x(r))\,d[x,x]^{ij}(r).
\end{equation}

Via this relation we can reconstruct Dynkin's formula in this setting. By defining the operator $A$ as in \ref{2-1-dynkform} we have:

\begin{align}
    \Phi(s,x(s)) & = \Phi(t,x) + \int_t^s\Phi_s(r,x(r))\,dr \\
    & + \sum_{i=1}^n \left[\int_t^s \Phi_{x^i}(r,x(r))f_i(r,x(r))\,dr + \sum_{j=1}^n\int_t^s\Phi_{x^i}(r,x(r))\sigma_{ij}(r,x(r))\,dw^j(r)\right] \\
    & + \frac{1}{2}\sum_{i,j=1}^n \sum_{l=1}^n\int_t^s\Phi_{x^ix^j}(r,x(r))\sigma_{il}(r,x(r))\sigma_{jl}(r,x(r))\,dr \\
    & = \Phi(t,x) + \int_t^s \Phi_s(r,x(r)) + D_x\Phi\cdot f(r,x(r)) + \frac{1}{2}D^2_x\Phi\cdot a(r,x(r)) \,dr \\
    & + \int_t^s D_x\Phi\cdot\sigma(r,x(r)) \,dw(r) \\
    % & = \Phi(t,x) + \int_t^s A\Phi(r,x(r))\,dr + \int_t^s D_x\Phi\cdot\sigma(r,x(r)) \,dw(r)
\end{align}

but the last (stochastic) integral can be seen as a martingale. In particular, if we take $\Phi$ to have polynomial growth of some order $m$:

\begin{equation}
    \abs{\Phi(t,x)} \leq K(1+\abs{x}^m)\,\forall(t,x)\in\overline{Q}_0
\end{equation}

then $D_x\Phi\cdot\sigma\in\mathcal{L}^2(I_0)$, where:

\[\mathcal{L}(I) = \left\{x:I\times\Omega\rightarrow\Sigma\,|\,E\int_Ix(s)\,ds<\infty\right\}\]

and therefore its stochastic integral is a martingale (with respect to the canonical filtration associated to the Brownian motion $w$). 
Therefore, if we take the (conditional) expectation:

\begin{equation}
    E_{tx}\Phi(s,x(s)) = \Phi(t,x) + E_{tx}\int_t^s \Phi_s(r,x(r)) + D_x\Phi\cdot f(r,x(r)) + \frac{1}{2}D^2_x\Phi\cdot a(r,x(r))\,dr.
\end{equation}

It is now natural to define the operator $A:C_p^{1,2}(\overline{Q}_0)\rightarrow\R$ as:

\begin{equation}\label{2-1-newdefA}
    A\Phi(r,x(r)) = \Phi_s(r,x(r)) + D_x\Phi\cdot f(r,x(r)) + \frac{1}{2}D^2_x\Phi\cdot a(r,x(r))
\end{equation}

where $C_p^{1,2}(I)$ is the family of functions $g$ from $I$ into $\R$ such that $g,g_s,g_{x_i},g_{x_ix_j}$ are continuous and with polynomial growth.

\begin{remark}
    Be careful that the stochastic integral:

    \[\int_t^s D_x\Phi\cdot\sigma(r,x(r))\,dw(r)\]

    is a martingale because $x$ satisfies:

    \[E_{tx}\abs{x(r)}^m\leq C_m(1+\abs{x}^m)\,\forall r\in I_0\]

    as it is solution of the SDE \ref{2-1-SDEdef}.\footnote{This is a standard result in SDE theory.}
\end{remark}

Consequently the generator $G$ of the time-homogeneous case is defined as:

\begin{equation}
    G\Phi(x) = -\frac{1}{2}\sum_{i,j=1}^n a_{ij}(x)\Phi_{x_ix_j}(x) - \sum_{i=1}^n f_i(x)\Phi_{x_i}(x)
\end{equation}


THEN CONTROLLED MARKOV PROCESSES.

I HAVE TO SEARCH FOR A MORE GENERAL VERSION OF EXISTENCE THEOREM for markov control policies.

THEN DYNAMIC PROGRAMMING APPROACH AND VERIFICATION THEOREM.

THEN EXAMPLE IF TIME ALLOWS.

THEN REREAD, REPEAT. I DONT THINK I'LL DO SLIDES. MAYBE THEY CAN BE A GOOD ASSET. LESS INFO, LESS THINGS TO KNOW PERFECTLY. BUT I HAVE TO KNOW EVERYTHING!!!

\end{document}