\section{Markov control processes}

So far we talked about Markov processes without specifying any kind of control. A control process in any 
stochastic process $u:\Omega\rightarrow U$, where $U$ is the control space, that influences the evolution of the random process $x$. 
Formally, let $Q=I_0\times O$ and $u$ as before and define:

\begin{equation}\label{2-1-controlledSDE}
    \begin{cases}
        dx(r) = f(r,x(r),u(r))dr + \sigma(r,x(r),u(r))dw(r)\,r\in I_0 \\
        x(t) = x
    \end{cases}
\end{equation}

where $U\subset\R^m$ closed, $f,\sigma\in C(\overline{Q}_0\times U)$, $f(\cdot,\cdot,v),\sigma(\cdot,\cdot,v)$ belong to $C^1(\overline{Q}_0)$ for all $v\in U$, 
such that there exists $C>0$ such that:

\begin{align}
        \abs{f_t}+\abs{f_x} \leq C,\,\abs{\sigma_t}+\abs{\sigma_x}\leq C \\
        \abs{f(t,x,v)}\leq C(1+\abs{x}+\abs{v}) \\
        \abs{\sigma(t,x,v)}\leq C(1+\abs{x}+\abs{v})
\end{align}


We can relax the assumption by imposing Lipschitz condition on $t$ and $x$ for every fixed $v$. Furthermore,
we assume $u$ to be \textit{admissible}, that is:

\begin{equation}\label{2-1-condoncontrol}
    E\int_t^{t_1} \abs{u(s)}^m\,ds <\infty\,\forall m\in\mathbb{N}.
\end{equation}

It is implied by $U$ being compact. Under these hypotheses, equation \ref{2-1-controlledSDE} has a unique (indistinguishable) solution. 
Where does optimality play its role? We define running and terminal costs $L,\Psi$, both continuous and satisfying:

\begin{align}
    \abs{L(s,x,v)} \leq C(1+\abs{x}^k+\abs{v}^k) \\
    \abs{\Psi(s,x)} \leq C(1+\abs{x}^k)
\end{align}

for suitable $C,k>0$. We also define $\tau$ to be the exit time of $(s,x(s))$ from $Q$. We define:

\begin{equation}
    J(t,x;u) = E_{tx}\left\{\int_t^{\tau} L(s,x(s),u(s))\,ds + \Psi(\tau,x(\tau))\right\}
\end{equation}

for every initial condition $(t,x)\in Q$ and control $u$. We aim to minimize this criterion, that is:

\[\inf_{u\in\mathcal{U}}J(t,x;u).\]

This formulation is not mathematically formal enough, let us restate it. We begin by defining an infimum criterion with respect to a probability space, or more 
formally a \textit{probability system}, and then we'll take the infimum over all probability systems. 

\begin{definition}
    A reference probability system is a tuple $(\Omega,\{\mathcal{F}_s\},P,\omega)$ such that:
    
    \begin{enumerate}[label=\alph*)]
        \item $\nu=(\Omega,\mathcal{F}_{t_1},P)$ is a probability space
        \item $\{\mathcal{F}_s\}$ is a filtration on $\Omega$
        \item $w$ is an $\mathcal{F}$-adapted Brownian motion on $[t,t_1]$.
    \end{enumerate}

    We denote with $\mathcal{A}_{t\nu}$ the collection of all $\mathcal{F}$ progressively measurable (that is $\mathcal{B}([t,s])\times\mathcal{F}_s$-adapted), 
    $U$ valued processes $u$ such that condition \ref{2-1-condoncontrol} holds on $[t,t_1]$.
\end{definition}

We define:

\begin{equation}\label{2-1-systemopt}
    V_{\nu} = \inf_{u\in\mathcal{A}_{t\nu}} J(t,x;u)
\end{equation}

while we define:

\begin{equation}\label{2-1-opt}
    V_{PM} = \inf_{\nu} V_{\nu}.
\end{equation}

Equation \ref{2-1-systemopt} and \label{2-1-opt} respectively define $\nu$-\textit{optimality} and \textit{optimality} for those control that 
satisfy them. We adapt the definition of operator $A$ to this situation by defining for every element of the control space $v$ the functional:

\begin{equation}\label{2-1-defAwithv}
    A^{v}\Phi = \Phi_t + \sum_{i=1}^nf_i(t,x,v)\Phi_{x_i} + \frac{1}{2}\sum_{i,j=1}^n a_{ij}(t,x,v)\Phi_{x_ix_j}, \Phi\in C^{1,2}_p(\overline{Q}_0)
\end{equation}

where $a=\sigma\sigma'$. As we did in the determinist case, we provide a heuristic derivation of the Hamilton-Jacobi-Bellman equation (the verification theorem), 
and then we'll formally prove it. Let us suppose that $O=\R^n$, then $J$ is:

\begin{equation}
    J(t,x;u) = \int_t^{t_1} L(s,x(s),u(s))\,ds + \Phi(t_1,x(t_1)).
\end{equation}

By the dynamic programming principle for every $h<t_1-t$:

\[V(t,x) = \inf_{u\in\mathcal{A}}E_{tx}\left\{\int_t^{t+h} L(s,x(s),u(s))\,ds + V(t+h,x(t+h))\right\}.\]

If we take the constant control $u\equiv v$ then by Dynkin's formula we get:

\begin{align}
    0 & \leq E_{tx}V(t+h,x(t+h))-V(t,x) + E_{tx}\int_t^{t+h} L(s,x(s),v)\,ds \\
    & = E_{tx}\int_t^{t+h} A^vV(s,x(s))\,ds + E_{tx}\int_t^{t+h} L(s,x(s),v)\,ds
\end{align}

dividing by $h$ and taking the limit for $h\to0^+$:

\[0 \leq A^vV(t,x) + L(t,x,v).\]

If we take $u^{\ast}$ to be optimal, then equality holds:

\[A^{u^{\ast}}V(t,x) + L(t,x,u^{\ast}(t)) = 0.\]

We now present the verification theorem rigorously. Let us define the Hamiltonian for this problem. For every $(t,x)\in\overline{Q}_0$, 
$p\in\R^n$ and $A\in\mathcal{S}_+^n$ (set of symmetric, non-negative definite $n\times n$ matrices) we define:

\begin{equation}\label{2-1-defhamilt}
    \mathcal{H}(t,x,p,A) =  \sup_{v\in U}\left\{-f(t,x,v)\cdot p - \frac{1}{2}tr\left[a(t,x,v)\cdot A\right] - L(t,x,v)\right\}
\end{equation}

where for matrices $A,B\in\R^{n\times n}$:

\begin{equation}
    tr\left(AB\right) = \sum_{i,j=1}^n A_{ij}B_{ji}, 
\end{equation}

which is equal to $\sum_{i,j=1}^n A_{ij}B_{ij}$ for symmetric matrices.

We can now state the verification theorem using the Hamiltonian defined in \ref{2-1-defhamilt}.

\begin{theorem}
    Let $W\in C^{1,2}(Q)\cap C_p(\overline{Q})$ such that:

    \begin{align}\label{2-1-hamilcond}
        -\frac{\partial W}{\partial t} + \mathcal{H}(t,x,D_xW,D_x^2W) = 0, & \forall(t,x)\in Q \\
        V(t,x) = \Psi(t,x), & \forall(t,x)\in\partial Q.
    \end{align}

    Then:

    \begin{enumerate}
        \item for any system $\nu$, initial condition $(t,x)\in Q$ and any $u\in \mathcal{A}_{t\nu}$ then:
        
        \begin{equation}
            W(t,x) \leq J(t,x;u)
        \end{equation}

        \item If there exists $\nu^{\ast}=(\Omega^{\ast},\{\mathcal{F}_s^{\ast}\},P^{\ast},w^{\ast})$ and $u^{\ast}\in\mathcal{A}_{t\nu^{\ast}}$ such that:
        
        \begin{equation}
            u^{\ast}(s) \in \argmin_{v\in U}\left\{f(s,x^{\ast}(s),v)\cdot D_xW(s,x^{\ast}(s)) + \frac{1}{2}tr\left[a(s,x^{\ast}(s),v)\cdot D_x^2(s,x^{\ast}(s))\right] + L(s,x^{\ast}(s),v)\right\}
        \end{equation}

        for almost all $(s,\omega)\in[t,\tau^{\ast}]\times\Omega^{\ast}$, then:

        \begin{equation}
            V_{PM}(t,x) = J(t,x;u^{\ast}).
        \end{equation}
    \end{enumerate}

    \begin{proof}
        We assume $O$ to bounded and $W\in C^{1,2}(\overline{Q})$. Because of \ref{2-1-hamilcond} for all $s\in[t,\tau]$:

        \begin{equation}\label{2-1-proofver-keyeq}
            0\leq A^{u(s)}W(s,x(s)) + L(s,x(s),u(s)).
        \end{equation}

        Because of Ito:

        \begin{equation}
            W(\tau,x(\tau)) - W(t,x) = \int_t^{\tau} A^{u(s)}W(s,x(s))\,ds + \int_t^{\tau} D_x\Phi(s,x(s))\cdot \sigma(s,x(s),u(s))\,dw(s).
        \end{equation}

        Because of estimates on SDE solution the last stochastic integral is a $\mathcal{F}_s$-martingale. Then if we take the expectation $E_{tx}$ we get:

        \begin{align}
            0 & \leq E_{tx} \int_t^{\tau}A^{u(s)}W(s,x(s))\,ds + E_{tx}\int_t^{\tau} L(s,x(s),u(s))\,ds \\
            & = E_{tx}\left(W(\tau,x(\tau)) - W(t,x)\right) - E_{tx}\int_t^{\tau} D_x\Phi(s,x(s))\cdot \sigma(s,x(s),u(s))\,dw(s) \\
            & + E_{tx}\int_t^{\tau} L(s,x(s),u(s))\,ds \\
            & = -W(t,x) + E_{tx}\left\{\int_t^{\tau} L(s,x(s),u(s))\,ds + W(\tau,x(\tau))\right\} \\
            & = -W(t,x) + J(t,x;u).
        \end{align}

        If $O$ is unbounded we define for every $\rho>0$ such that $\rho^{-1}<t_1-t_0$ the set:

        \begin{equation}
            O_{\rho} = O\cap\left\{\abs{x}<\rho\,|\,d(x,\partial O)>\frac{1}{\rho}\right\},\, Q_{\rho} = [t_0,t_1-\rho^{-1}]\times O_{\rho}
        \end{equation}

        and $\tau_{\rho}$ the exit time from $Q_{\rho}$. Then $Q_{\rho}$ is bounded, and $W\in C^{1,2}(\overline{Q}_{\rho})$, then:

        \begin{equation}
            W(t,x) \leq E_{tx}\left\{\int_t^{\tau_{\rho}} L(s,x(s),u(s))\,ds + W(\tau_{\rho},x(\tau_{\rho}))\right\}.
        \end{equation}

        We now take $\rho\to+\infty$ and get the thesis. We have convergence in probability for $\tau_{\rho}\xrightarrow{\rho\to+\infty}\tau$. 
        We prove uniform integrability of the rhs and therefore get $L^1$ convergence. We have:
        
        \begin{align}
            E_{tx}\int_t^{\tau_{\rho}} \abs{L(s,x(s),u(s))}\,ds & \leq E_{tx}\int_t^{t_1}\abs{L(s,x(s),u(s))}\,ds \\
            & \leq E_{tx}\int_t^{t_1}\left(1+\abs{x(s)}^k+\abs{u(s)}^k\right)\,ds < +\infty    
        \end{align}
        
        because $u$ is admissible and estimates on SDE solutions. While we have:

        \begin{align}
            E_{tx}\abs{W(\tau_{\rho},x(\tau_{\rho}))}^{\alpha} & \leq KE_{\tx}\left(1+\abs{x(\tau_{\rho})}^k\right)^{\alpha} \\
            & \leq 2^{\alpha-1}K\left(1+E_{tx}\norm{x}^{\alpha k}\right) \leq C
        \end{align}

        for $\alpha>\frac{1}{k}$ and estimates on SDE solutions. Therefore we get:

        \begin{equation}
            \lim_{\rho\to+\infty} E_{tx}\left\{\int_t^{\tau_{\rho}} L(s,x(s),u(s))\,ds + W(\tau_{\rho},x(\tau_{\rho}))\right\} = E_{tx}\left\{\int_t^{\tau} L(s,x(s),u(s))\,ds + W(\tau,x(\tau))\right\}.
        \end{equation}

        Part $b)$ comes from equality in equation \ref{2-1-proofver-keyeq}.
    \end{proof}
\end{theorem}

% THEN DYNAMIC PROGRAMMING APPROACH AND VERIFICATION THEOREM: i present the formal derivation via $A$. Then the theorem. 
% THEN CONTROLLED MARKOV PROCESSES.

% I HAVE TO SEARCH FOR A MORE GENERAL VERSION OF EXISTENCE THEOREM for markov control policies.


% THEN EXAMPLE IF TIME ALLOWS.

% THEN REREAD, REPEAT. I DONT THINK I'LL DO SLIDES. MAYBE THEY CAN BE A GOOD ASSET. LESS INFO, LESS THINGS TO KNOW PERFECTLY. BUT I HAVE TO KNOW EVERYTHING!!!