\section{Introduction}

As previously mentioned, in many instances the value function arising from an optimal control problem may fail to be continuously 
differentiable. If that happens the derivation of the Hamilton-Jacobi equation is no longer valid, but more importantly 
the notion of classical solution to it does hold anymore. Therefore, we have to weaken the notion of solution in order to get 
a consistent and unique solution to the dynamic programming equation for non-differentiable value functions. The \textit{viscosity solution} 
is exactly what we are searching for. It arises from a standard procedure called vanishing viscosity, which allows us to compute the 
solution of a fully non-linear first order PDE as the limiting solution of quasilinear parabolic PDEs, obtained via infinitesimal 
perturbations of second order derivatives. 

\subsection{Non-differentiable value functions}

Let us consider the calculus of variation problem:

\begin{equation}
    \inf_{x\in Lip([0,1];[-1,1])} \int_t^{t_1} 1 + \frac{1}{4}(\dot{x}(s))^2 \,ds,
\end{equation}

where $Lip(I;U)$ is the collection of Lipschitz continuous functions from $I$ to $U$. The Hamiltonian related to 
this problem is:

\[H(t,x,p) = \max_{v\in [-1,1]} \left\{-v\cdot p - 1 - \frac{1}{4}v^2\right\}.\]

We can explicitly compute the Hamiltonian and get:

\[H(t,x,p)=p^2-1.\]

Then the Hamilton-Jacobi equations read:

\[\begin{cases}
    \dot{x}^{\ast}(s) = -H_p(s,x^{\ast}(s),p^{\ast}(s)) =  2p^{\ast}(s)\\
    \dot{p}^{\ast}(s) = H_x(s,x^{\ast}(s),p^{\ast}(s)) = 0,
\end{cases}\]

therefore, we get:

\[\dot{x}(s)^{\ast} = 2p^{\ast},\,s\in[0,1],\]

for some $p^{\ast}\in\R$. We now compute the exit time of $(s,x(s))=(s,2(s-t)p^{\ast}+x)$ with initial data $(t,x)$. If $p=0$ then:

\[\tau=1,\,\abs{x}<1.\]

If $p>0$ then $x(s)=2(s-t)p+x$ is increasing, which implies that the system is going to exit from the right boundary, that is from $x(s)=1$, and if 
that happens before time $s=1$ the exit time will be determined by:

\[2(s-t)p+x=1\Rightarrow s=t+\frac{1-x}{2p}.\]

$x(s)=1$ for $s<1$ if:

\[2(1-t)p+x\geq1\Rightarrow p\geq t+\frac{1-x}{2p},\]

therefore:

\[\tau=\begin{cases}
    1 & p\geq t+\frac{1-x}{2p} \\
    t+\frac{1-x}{2p} & p> t+\frac{1-x}{2p}.
\end{cases}\]

Analogously, if $p<0$:

\[\tau=\begin{cases}
    1 & p\leq t-\frac{1+x}{2p} \\
    t-\frac{1+x}{2p} & p< t-\frac{1+x}{2p}.
\end{cases}.\]

We now solve:

\[\inf_{p\in\R} \int_t^{\tau}1+p^2\,ds=\inf_{p\in\R}(1+p^2)(\tau-t)=\begin{cases}
    (1+p^2)(1-t), & p=0\text{ or }p>0\land p\geq \frac{1-x}{2(1-t)}\text{ or }p<0\land p\leq\frac{-1-x}{2(1-t)}\\
    (1+p^2)\frac{1-x}{2p}, & p>0\land p\geq \frac{1-x}{2(1-t)} \\
    -(1+p^2)\frac{1+x}{2p}, & p\leq\frac{-1-x}{2(1-t)},
\end{cases}\]

which is solved as follows:

\[V(t,x)=\begin{cases}
    1-t & \abs{x}\leq t \\
    1-t& \abs{x}\geq t,
\end{cases}\]

which is continuous on the whole space, but clearly no differentiable in $\abs{x}=t$.