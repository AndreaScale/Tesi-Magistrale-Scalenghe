\section{Proof of Pontryagin's principle}

We will show the maximum Pontryagin's principle in the simple context of no running cost. 
A cleaver reconstruction of the non-zero running cost problem as a zero running cost one 
will enlarge the thesis to this situation. The first concept we will need is the variation 
of a control.

\begin{definition}
    Given $u\in\mathcal{U}$. For $\epsilon,r>0$ such that $0<r-\epsilon<r$ and $a\in U$ 
    we define the \textit{simple variation} $u_{\epsilon}\in\mathcal{U}$ such that:

    \begin{equation}\label{1-proofpontry-defvar}
        u_{\epsilon}(t) = \begin{cases}
            a & s\in(r-\epsilon,r) \\
            u(s) & s\notin(r-\epsilon,r)
        \end{cases}
    \end{equation}
\end{definition}

By defining the matrix $A:[0,+\infty)\rightarrow\R^{n\times n}: s\mapsto D_xf(s,x(s),u(s))$ we state 
the following lemma.

\begin{lemma}
    Let $x_{\epsilon}$ be solution of:

    \begin{equation}\label{1-proofpontry-dynamprob}
        \begin{cases}
            \dot{x}_{\epsilon}(s) = f(s,x_{\epsilon}(s),u_{\epsilon}(s)) & s\in[t,t_1] \\
            x_{\epsilon}(t) = x
        \end{cases}
    \end{equation}

    Then the solution is:

    \begin{equation}\label{1-proofpontry-solvar}
        x_{\epsilon}(s) = x(s) + \epsilon y(s) + o(\epsilon)\, \epsilon\to0
    \end{equation}

    Where $y\equiv0$ on $[t,r]$ and:

    \begin{equation}
        \begin{cases}
            \dot{y}(s) = A(s)y(s) & s\in[r,t_1]\\
            y(r) = y^r
        \end{cases}
    \end{equation}

    With $y^r=f(x(r),a)-f(x(r),u(r))$.

    \begin{proof}
        Let us divide the proof in three cases.

        \begin{itemize}
            \item $s\in[t,r-\epsilon]$: then $y(t)=0$ and $u_{\epsilon}(t)=u(t)$, therefore:
            
            \[x_{\epsilon}(t)=x(t)=x(t)+\epsilon y(t)+o(\epsilon)\]

            \item $s\in(r-\epsilon,r)$: then we have:
            
            \[x_{\epsilon}(s)-x(s) = \int_{r-\epsilon}^s f(w,x_{\epsilon}(w),u_{\epsilon}(w))-f(w,x(w),u(w)) \,dw + o(\epsilon)\]

            Which is a little \textit{o} of $\epsilon$ (because $f$ is continuous). 

            \item $s\in[r,t_1]$: from before if $s=r$ then:
            
            \begin{align}
                x_{\epsilon}(r)-x(r) & = \int_{r-\epsilon}^r f(w,x_{\epsilon}(w),u_{\epsilon}(w))-f(w,x(w),u(w)) \,dw + o(\epsilon) \\
                & = \lim_{w\to s}[f(w,x_{\epsilon}(w),u_{\epsilon}(w))-f(w,x(w),u(w))]\epsilon+ o(\epsilon) \\
                & = y^s\epsilon + o(\epsilon)
            \end{align}

            \color{red}{If $s>r$ then:}


        \end{itemize}
    \end{proof}
\end{lemma}

Let us now prove Pontryagin's principle with no running cost. The payoff functional is:

\begin{equation}\label{1-proofpontry-norunfunct}
    J(t,x;u) = \psi(x(t_1))
\end{equation}

And therefore the Hamiltonian is:

\begin{equation}\label{1-proofpontry-norunham}
    H(s,x,u,p) = - f(s,x,u)\cdot p 
\end{equation}

\begin{theorem}\label{1-proofpontry-theo}
    There exists a function $p^{\ast}:[t,t_1]\rightarrow \R^n$ such that:

    \begin{equation}
        \dot{p}^{\ast}(s) = -D_x H(s,x^{\ast}(s),u^{\ast}(s),p^{\ast}(s))\, s\in[t,t_1]
    \end{equation}

    together with the maximization:

    \begin{equation}
        H(s,x^{\ast}(s),u^{\ast}(s),p^{\ast}(s)) = \sup_{v\in U} H(s,x^{\ast}(s),v,p^{\ast}(s))
    \end{equation}

    and the trasversality condition:

    \begin{equation}
        p^{\ast}(t_1) = D\psi(x(t_1))
    \end{equation}

    \begin{proof}
        Let us drop all the $^{\ast}$. Let $p$ be the unique solution of:

        \begin{equation}
            \begin{cases}
                \dot{p}(s) = - A'(s)\cdot p(s) & s\in[t,t_1] \\
                p(t_1) = D\psi(x(t_1))
            \end{cases}
        \end{equation}

        It exists and is unique because the latter is a linear differential equation with integrable coefficient. 
        We already satisfy the trasversality condition and the adjoint dynamics. We prove the 
        maximization principle. Let $v\in U$. We define the variation $u_{\epsilon}$ for $\epsilon,r\in(t,t_1)$ as before.
        Since $\epsilon\mapsto  J(t,x;u_{\epsilon})$ for $\epsilon\in[0,1]$ has a maximum in $\epsilon=0$ we have:
        
        \begin{equation}\label{1-proofpontry-dervar0}
            \frac{d}{d\epsilon}J(t,x;u_{\epsilon})\leq0
        \end{equation}

        Computing the derivative, using \ref{1-proofpontry-defvar}:

        \begin{align}
            \frac{d}{d\epsilon}J(t,x;u_{\epsilon})\big|_{\epsilon=0} & = \frac{d}{d\epsilon}\psi(x_{\epsilon}(t_1))\big|_{\epsilon=0} \\
            & = \frac{d}{d\epsilon}\psi(x(t_1) + \epsilon y(t_1) + o(\epsilon)) = D\psi(x(t_1))\cdot y(t_1) \\
            & = p(t_1)\cdot y(t_1) = p(r)\cdot[f(r,x(r),a)-f(r,x(r),u(r))]
        \end{align}

        Where the last equality comes from:

        \begin{align*}
            \frac{d}{ds}\left(p(s)\cdot y(s)\right) & = \dot{p}(s)\cdot y(s) + p(s)\cdot \dot{y}(s) \\
            & = -A'(s)\cdot p(s)\cdot y(s) + p(s) \cdot A(s)\cdot y(s) = 0
        \end{align*}

        Therefore, by plugging into \ref{1-proofpontrydervar0} we get:

        \[0\geq p(r)\cdot[f(r,x(r),a)-f(r,x(r),u(r))]\]

        Which implies:

        \[H(r,x(r),v,p(r)) = f(r,x(r),v) \cdot p(r) \leq f(r,x(r),u(r)) \cdot p(r) = H(r,x(r),u(r),p(r))\]
    \end{proof}
\end{theorem}

Given Pontryagin's principle for no running cost problems we can extend the result to the general case:

\begin{equation}
    J(t,x;u) = \int_t^{t_1} L(s,x(s),u(s)) \,ds + \psi(x(t_1))
\end{equation}

Where the Hamiltonian is:

\begin{equation}
    H(s,x,u,p) = f(s,x,u)\cdot p + L(s,x,u)
\end{equation}

Indeed, theorem \ref{1-proofpontry-theo} holds also under these conditions. We rewrite the problem as it has no 
running cost and then apply the theorem. Let us define $x^{n+1}$ as:

% \begin{equation}
%     \begin{cases}
%         \dot{x}^{n+1}(s) = L(s,x^{n+1}(s),u(s)) & s\in [t,t_1] \\
%         x^{n+1}(0) = 0 
%     \end{cases}
% \end{equation}

\begin{equation}
    x^{n+1}(s) = \int_t^s L(w,x(w),u(w)) \,dw
\end{equation}

Then by defining $\overline{f}, \overline{g}, \overline{x},\overline{x}(s)$ as:

\begin{equation}
    \overline{f}(s,x,u)=\begin{bmatrix}
        f(s,x,u) \\
        L(s,x,u)
    \end{bmatrix},\,\overline{x}=\begin{bmatrix}
        x \\
        0
    \end{bmatrix},\,\overline{x}(s)=\begin{bmatrix}
        x(s) \\
        x^{n+1}(s)
    \end{bmatrix},\,\overline{g}(\overline{x}(t_1)) = g(x(t_1)) + x^{n+1}(t_1)
\end{equation}

Thus, the problem has no running cost. We can apply the theorem and, noticing that 
$p^{n+1}\equiv1$ we get the thesis.