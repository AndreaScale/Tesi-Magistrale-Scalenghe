\section{Pontryagin's principle}

We will use the notion of differentiability of the value function. 
The classical notion of differentiability which we use is the following.

\begin{definition}
    V is differentiable in $(t,x)$ if there exists $V_t(t,x),V_x(t,x)\in\R$ such that:

    \begin{equation}
        \lim_{(h,k)\to(0,0)} \frac{1}{\abs{h}+\abs{k}}\abs{V(t+h,x+k)-V(t,x)-V_t(t,x)\cdot h-V_x(t,x)\cdot k}=0
    \end{equation}
\end{definition}

Differentiability is somewhat a strong hypothesis, but it allows us to prove the folllwing
proposition.

\begin{theorem}
    Let $V$ be differentiable in $(t,x)\in Q$ and $u^{\ast}$ an optimal control
    such that $u^{\ast}\xrightarrow{s\to t}v$, then:

    \begin{equation}\label{1-3-dynamicprogrammeq}
        V_t(t,x) + L(t,x,v) + f(t,x,v)\cdot D_xV(t,x) = 0
    \end{equation}

    \begin{proof}
        Let $h>0$ s.t. $t+h<\tau$, then by \ref{1-2-propriformulation} we have:

        \[ V(t,x) = \int_t^{t+h} L(s,x(s),u^{\ast}(s)) \,ds + V(t+h,x(t+h))\]

        But because of differentiability we have:

        \[\lim_{h\to0}\frac{1}{\abs{h}}\abs{V(t+h,x(t+h))-V(t,x(t))} = V_t(t,x) + f(t,x,v)\cdot D_xV(t,x)\]

        Then we get:

        \begin{align}
        L(t,x,v) & = \lim_{h\to0} \frac{1}{\abs{h}} \int_t^{t+h} L(s,x(s),u(s)) \,ds = \lim_{h\to0} \frac{1}{\abs{h}}\abs{V(t+h,x(t+h))-V(t,x(t))} \\
        & = V_t(t,x) + f(t,x,v)\cdot D_xV(t,x)
        \end{align}
    \end{proof}
\end{theorem}