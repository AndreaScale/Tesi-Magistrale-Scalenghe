\documentclass[12pt,a4paper]{book}
% \usepackage{times}
\usepackage[a4paper, margin=0.6in]{geometry}
\usepackage[utf8]{inputenc}
\usepackage[english]{babel}
\selectlanguage{english}
\usepackage{amsmath,amsthm,amssymb,fge,mathrsfs}
\usepackage{mathtools}
\usepackage{hyperref}
\usepackage{makeidx} %per fare l'indice
\usepackage{faktor} %per i quozienti

%tikz (pacchetto per i disegni)
\usepackage{tikz}
\usetikzlibrary{arrows}
\usetikzlibrary{tikzmark}
\usetikzlibrary{calc} %per poter fare i calcoli
\usetikzlibrary{arrows.meta} %per usare veri tipi di frecce
\usetikzlibrary{calc,patterns,angles,quotes} %per disegnare gli angoli
\usetikzlibrary{decorations} %per i grafici orientati (con sopra le frecce)
\usetikzlibrary{decorations.markings}
\usetikzlibrary{backgrounds} %per il colore sullo sfondo delle immagini
\usetikzlibrary{shapes.geometric} %per i triangoli negli alberi

\usepackage{pgfplots}
\pgfplotsset{compat=newest}
\usepgfplotslibrary{fillbetween} %per colorare le aree comprese tra due grafici

\usepackage{caption} %per la caption sotto alle immagini
\usepackage[tracking=true]{microtype} %per avvicinare il testo
\usepackage{enumitem} %per modificare agilmente gli elenchi
\usepackage{esvect} %per le freccette di vettore sui caratteri
\usepackage{stmaryrd} %fulmine per il simbolo dell'assurdo
\usepackage{relsize} %per fare i simboli più grandi
\usepackage{mathabx} %per il simbolo di dotminus
\usepackage{esint} %per l'integrale tagliato (fint)

\usepackage[customcolors]{hf-tikz} %per evidenziare pezzi di matrice

\usepackage{verbatim} %per commentare un blocco

\usepackage{multirow} %per la tabella
\usepackage{array} %per lo spazio nella tabella

%Libreria per gli alberi
\usepackage[linguistics]{forest}

%Librerie aggiunte
\usepackage{multicol} %per gli elenchi puntati su più colonne/righe
\usepackage{multirow}

%pacchetto per il prodotto interno
\usepackage{physics}

%Libreria per i subfiles

\usepackage{subfiles} %meglio caricarla per ultima

\usepackage{biblatex}
\addbibresource{biblio.bib}
%usiamo paragraph come nuova sottosottosottosezione 
\makeatletter
\renewcommand\paragraph{\@startsection{paragraph}{4}{\z@}%
            {-2.5ex\@plus -1ex \@minus -.25ex}%
            {1.25ex \@plus .25ex}%
            {\normalfont\normalsize\bfseries}}
\makeatother
\setcounter{secnumdepth}{4} % how many sectioning levels to assign numbers to
\setcounter{tocdepth}{4}    % how many sectioning levels to show in ToC

\newtheorem{theorem}{Theorem}
\numberwithin{theorem}{section}
\newtheorem*{theorem*}{Theorem}
\newtheorem{corollary}[theorem]{Corollary}
\newtheorem{proposition}[theorem]{Proposition}
\newtheorem{lemma}[theorem]{Lemma}
\newtheorem*{lemma*}{Lemma}

\newtheorem{definition}{Definition}
\numberwithin{definition}{section}

\newtheorem*{remark}{Remark}

\newtheorem{example}{Example}
\numberwithin{example}{section}

\newtheorem{exercise}{Exercise}
\numberwithin{exercise}{section}

\renewcommand{\proofname}{\proof}

\renewcommand{\thefootnote}{[\arabic{footnote}]}% Modify footnote globally

%%%simboli speciali
\newcommand{\N}{\mathbb{N}}
\newcommand{\Z}{\mathbb{Z}}
\newcommand{\Q}{\mathbb{Q}}
\newcommand{\R}{\mathbb{R}}
\newcommand{\C}{\mathbb{C}}

\newcommand{\CO}{\mathcal{C}}

\DeclareMathOperator{\Dom}{Dom}
\DeclareMathOperator{\Ima}{Im}

\DeclareMathOperator{\mcd}{MCD}
\DeclareMathOperator{\mcm}{mcm}
\DeclareMathOperator{\supp}{supp}
\DeclareMathOperator{\spanlin}{span}

%\DeclarePairedDelimiter\abs{\lvert}{\rvert}
%\DeclarePairedDelimiter\norm{\lVert}{\rVert}

%Necessario per far si che la dimensione del valore assoluto e della norma si adatti all'aromento passato
%\makeatletter
%\let\oldabs\abs
%\def\abs{\@ifstar{\oldabs}{\oldabs*}}
%
%\let\oldnorm\norm
%\def\norm{\@ifstar{\oldnorm}{\oldnorm*}}
%\makeatother

%rinomino il comando per il prodotto interno
\let\pint\braket
%comando per il prodotto scalare
\newcommand{\pscal}[2]{\left\langle #1, #2 \right\rangle}

%Spaziatura per i quantificatori
\let\oldforall\forall
\renewcommand{\forall}{\; \oldforall \;}
\let\oldexists\exists
\renewcommand{\exists}{\; \oldexists \;}

%freccia per la convergenza debole
\newcommand{\longrightharpoonup}{\xrightharpoonup{\phantom{AB}}}

\newcommand{\warrow}{\xrightharpoonup{\:\, w \:\, }}
\newcommand{\sarrow}{\overset{s}{\longrightarrow}}
\newcommand{\wastarrow}{\xrightharpoonup{ w^* }}

\DeclareMathOperator*{\argmax}{arg\,max}
\DeclareMathOperator*{\argmin}{arg\,min}

\setlength{\parindent}{0in} 
\title{Title}
\author{Andrea Scalenghe}
\date{March 2024}



\begin{document}

%I want to understand how to proceed also in view of stochastics

\section{Introduction}

We start our dissertation by studying deterministic dynamic control. The system we aim to control is governed by ordinary differential equations. 

Short description of what is done in this chapter.

\section{Motivating example, discounted cost infinite horizon} %the idea is to use a working example that we borrow all over the dissertation. Starting deterministic we get stochastic and generalize it




\section{Finite horizon}


Let us consider a finite interval $I=[t,t_1]\subset\R$ as the operating time of the system. At each time $s\in I$ the system is described by $x(s)\in O\subseteq \R^n$ and controlled by $u(s)\in U\subseteq\R^n$
called control space. The system is described by:

\begin{equation}\label{1-1-syst1}
    \begin{cases}
        \dot{x}(s) = f(s,x(s),u(s)) & s\in I \\
        x(t) = x
    \end{cases}
\end{equation}

For a given $x\in O$ and suitable $f:\overline{Q}\times U \rightarrow \R^m$, where $Q_0=[t,t_1)\times O$. That is we impose $f\in C(\overline{Q}\times U)$ and the existence of $K_{\rho}>0$ for all $\rho>0$:

\begin{equation}\label{1-1-lipsch}
    \abs{f(t,x,v)-f(t,y,v)} \leq K_{\rho}\abs{x-y}
\end{equation}

For all $t\in I$, $x,y\in O$ and $v\in U$ such that $\abs{v}\leq\rho$. Under this conditions the system \ref*{1-1-syst1} has a unique solution. 
Controls $u(\cdot)$ are assumed to be in the set $L^{\infty}\left([t,t_1];U\right)$. We will soon specify more about the set of controls.

We have described a control problem. The concept of optimality is related some value function, specified by payoffs (or costs) associated to the system's states.
Let $L\in C(\overline{Q}\times U)$ be the \textit{running cost} and $\Psi\in C(I\times O)$ the \textit{terminal cost} defined as:

\begin{equation}\label{1-1-deftermcost}
    \Psi(t,x) = \begin{cases}
        g(t,x) & \text{if } (t,x)\in [t,t_1)\times O \\
        \psi(x) & \text{if } (t,x)\in \{t_1\}\times O  
    \end{cases}
\end{equation}

We define the \textit{payoff} $J$ as:

\begin{equation}\label{1-1-payoff1}
    J(t,x;u) = \int_t^{\tau}L(s,x(s),u(s)) \,ds + \Psi(\tau, x(\tau))
\end{equation}

Where $\tau$ is the exit time of $(s,x(s))$ from $\overline{Q}$, that is:

\begin{equation}\label{1-1-taudef}
    \tau = \begin{cases}
        \inf\{s\in [t,t_1)\,|\, x(s)\notin \overline{O}\} & \text{if } \exists s\in [t,t_1): x(s)\notin\overline{O} \\
        t_1 & \text{if }  x(s)\in\overline{O}\,\forall s\in [t,t_1)
    \end{cases}
\end{equation}

Then a control $u^{\ast}(\cdot)$ is \textit{optimal} if:

\begin{equation}\label{1-1-optimalcondition}
    J(t,x;u^{\ast}) \leq J(t,x;u) \quad \forall u\in L^{\infty}(I;U)
\end{equation}

Actually, we are being to generous with the control space. We have to impose a further condition on it, the \textit{switching condition}.
Let us assume that we have $u\in\mathcal{U}(t,x)$ and $u'\in\mathcal{U}(r,x(r))$ for $r\in[t,\tau]$. If we define:

\begin{equation}\label{1-1-switcond}
    \tilde{u}(s)=\begin{cases}
        u(s) & s\in[t,r) \\
        u'(s) & s\in[r,t_1]
    \end{cases}
\end{equation}

Then we impose:

\begin{equation}
    \tilde{u}_s\in\mathcal{U}(s,\tilde{x}(s)) \quad \forall s\in[t,\tilde{\tau}]
\end{equation}

Where $\tilde{x}$ is the solution to the control problem \ref*{1-1-syst1} with control $\tilde{u}$ and initial condition $x$, 
$\tilde{u}_s$ is the restriction of $\tilde{u}$ to $[s,t_1]$ and $\tilde{\tau}$ is the exit time of $(s,\tilde{x}(s))$ from $\overline{Q}$.
This condition assures that admissible controls can be replaced as the time evolves and the resulting control is still admissible.  %Finite Horizon problem

\section{Dynamic programming principle}

One way of tackling certain optimal control problems is via \textit{dynamic programming}. 
% The idea is to use a \textit{value function} as a tool to solve the control problem, finding sufficient and sometimes necessary conditions.
Let us define the \textit{value function}:

\begin{equation}\label{1-2-valuefunc}
    V(t,x)=\inf_{u\in \mathcal{U}(t,x)}J(t,x;u)
\end{equation}

For all $(t,x)\in\overline{Q}$. We get rid of the instance in which $V(t,x)=-\infty$ assuming $Q$ to be compact, or $L$ and $\Psi$ to be bounded below.
We aim at retrieving the argument which attains the infimum of \ref*{1-2-valuefunc}. In order to immerse this optimal control problem into 
a dynamic programming one we see the state of the system as the state of the variable and the control function as the decision function.
The basic idea behind dynamic programming techniques is to subdivide a problem into smaller problems, what does this mean in our context?
We will be able to find instantaneous the value function $V$ via a partial differential equation (PDE) called Hamilton-Jacobi-Bellman equation.

We start by stating and proving the following proposition, which provides us with an equivalent definition of the value function.

\begin{proposition}\label{1-2-propriformulation}
    For any $(t,x)\in\overline{Q}$ and any $r\in I$ then:

    \begin{equation}\label{1-2-riformulationvalue}
        V(t,x)=\inf_{u\in\mathcal{U}(t,x)}\left\{\int_t^{r\land\tau}L(s,x(s),u(s))\,ds+g(\tau,x(\tau))\chi_{\tau<r}+V(r,x(r))\chi_{r\leq\tau}\right\}
    \end{equation}

    \begin{proof}
        Value function less than rhs. If $r>\tau$ then $\tau<t_1$ and $\Psi(r\land\tau,x(r\land\tau))=g(\tau,x(\tau))$ and then \ref*{1-2-riformulationvalue}
        follows directly by definition. If $r\leq\tau$, let $\delta>0$ then there exists $u^1\in\mathcal{U}(r,x(r))$ such that:

        \[\int_r^{\tau^1}L(s,x^1(s),u^1(s))\,ds+\Psi(\tau^1,x^1(\tau^1))\leq V(r,x(r))+\delta\]

        Where $x^1$ is the state function corresponding to $u^1$ with initial condition $(r,x(r))$ and $\tau^1$ the first exit from $\overline{Q}$ of $(s,x^1(s))$.  
        By defining $\tilde{u}$ as for the switching condition \ref{1-1-switcond} we have $\tau^1=\tilde{\tau}$, because $\tau\geq r$ and then $\tilde{u}$ is $u^1$. Then:
        
        \begin{align*}
            V(t,x) & \leq V(t,x;\tilde{u}) \\
            & = \int_t^{\tilde{\tau}} L(s,\tilde{x}(s),\tilde{u}(s))\,ds+\Psi(\tilde{\tau},\tilde{x}(\tilde{\tau})) \\
            & = \int_t^{r} L(s,x(s),u(s))\,ds+\int_r^{\tau^1} L(s,x^1(s),u^1(s))\,ds+\Psi(\tau^1,x^1(\tau^1)) \\
            & \leq \int_t^{r} L(s,x(s),u(s))\,ds + V(r,x(r)) + \delta
        \end{align*}

    Since $\delta$ is arbitrary the first inequality is proved.
    
    \noindent Value function is bigger than rhs. Let $\delta>0$ and $U\in\mathcal{U}(t,x)$ such that:

    \[\int_r^{\tau}L(s,x(s),u(s))\,ds+\Psi(\tau,x(\tau))\leq V(t,x)+\delta\]

    Then: 

    \begin{align*}
        V(t,x) & \geq \int_r^{\tau}L(s,x(s),u(s))\,ds+\Psi(\tau,x(\tau)) - \delta \\
        % & = \int_t^{\tilde{\tau}} L(s,\tilde{x}(s),\tilde{u}(s))\,ds+\Psi(\tilde{\tau},\tilde{x}(\tilde{\tau})) \\
        & = \int_t^{r\land\tau} L(s,x(s),u(s))\,ds+\int_{r\land\tau}^{\tau} L(s,x(s),u(s))\,ds+\Psi(\tau,x(\tau)) -\delta\\
        % & \leq \int_t^{r} L(s,x(s),u(s))\,ds + V(r,x(r)) + \delta
        & = \int_t^{r\land\tau} L(s,x(s),u(s))\,ds+J(r,x(r))\chi_{r\leq\tau}+g(\tau,x(\tau))\chi_{\tau<r} - \delta\\
        & = \int_t^{r\land\tau} L(s,x(s),u(s))\,ds+V(r,x(r))\chi_{r\leq\tau}+g(\tau,x(\tau))\chi_{\tau<r} - \delta
    \end{align*}

    As $\delta$ is arbitrary we proved the proposition.
    \end{proof}
\end{proposition}


In the proof we used the concept of $\delta$-\textit{optimal control}, that is the control function $u\in\mathcal{U}(r,x(r))$ such that:

\[\int_r^{\tau^1}L(s,x^1(s),u^1(s))\,ds+\Psi(\tau^1,x^1(\tau^1))\leq V(r,x(r))+\delta.\]

This new representation allows us to find the so-called \textit{dynamic programming equation}. We have to impose that the value function 
is continuously differentiable, although this is not always the case. If differentiability fails, the notion of viscosity solution is needed. 

Let us first impose boundary conditions of the value function. Clearly if $t=t_1$ then:

\begin{equation}\label{1-2-boundcond1}
    V(t_1,x)=\psi(x)\,\forall x\in \overline{O}
\end{equation}

If $(t,x)\in [t_0,t_1)\times\partial O$ then the value function is $g$:

\begin{equation}\label{1-2-boundcond2}
    V(t, x) = g(t,x)
\end{equation}

Before stating the fundamental theorem which gives sufficient conditions 
for a solution to the optimal problem we follow a heuristic reasoning which will help our intuition. Under the hypothesis of continuous differentiability of the value function 
let us rewrite the dynamic programming principle as:

\begin{equation}
    \inf_{u\in\mathcal{U}}\left\{\frac{1}{h}\int_t^{(t+h)\land\tau} L(s,x(s),u(s))\,ds + \frac{1}{h}g(\tau,x(\tau))\chi_{\tau<t+h} + \frac{1}{h}\left[V(t+h,x(t+h))\chi_{\tau\geq t+h} - V(t,x)\right]\right\}=0
\end{equation}

Then if we formally let $h\to0$ we get:

\[\inf_{u\in\mathcal{U}}\left\{L(t,x(t),u(t)) + \partial_tV(t,x(t)) + D_xV(t,x(t))\cdot f(t,x(t),u(t))\right\}=0\]

Which can be rewritten as:

\begin{equation}\label{1-2-HJB1}
    -\frac{\partial}{\partial t}V(t,x) + H(t,x,D_xV(t,x))=0
\end{equation}

Where for $(t,x,p)\in \overline{Q}\times\R^n$ the Hamiltonian is defined as:

\begin{equation}\label{1-2-Hamiltonian1}
    H(t,x,p) = \sup_{v\in\R^n}\left\{-p\cdot f(t,x,v) - L(t,x,v)\right\}.
\end{equation}

Equation \ref{1-2-HJB1} turns out to be the main sufficient condition for the value function to be optimal.

Maybe only differentiability is needed (also for if, for only if we already know).

\begin{theorem}[Verification Theorem]\label{1-2-Verificationthe}
    Let $W\in C^1(\overline{Q})$ satisfy \ref{1-2-HJB1} and the boundary conditions \ref{1-2-boundcond1} and \ref{1-2-boundcond2} then:

    \[W(t,x)\leq V(t,x) \, \forall (t,x)\in\overline{Q}\]

    Moreover, there exists $u^{\ast}\in\mathcal{U}$ such that:

    \begin{equation}\label{1-2-uoptim}
        \begin{cases}
            L(s,x^{\ast}(s),u^{\ast}(s)) + f(s,x^{\ast},u^{\ast}(s))\cdot D_xW(s,x^{\ast}(s)) = - H(s,x^{\ast}(s),D_xW(s,x^{\ast}(s))) & \text{a.s. for } s\in[t,\tau^{\ast}] \\
            W(\tau^{\ast},x^{\ast}(\tau^{\ast})) = g(\tau^{\ast},x^{\ast}(\tau^{\ast})) & \text{if } \tau^{\ast}<t_1
        \end{cases}    
    \end{equation}

    if and only if $u^{\ast}$ is optimal and $W=V$.

    \begin{proof}
        Let $u\in\mathcal{U}$, then:

        \begin{align*}
            \Psi(\tau,x(\tau)) & = W(\tau,x(\tau)) = W(t,x(t)) + \int_t^{\tau}\frac{d}{ds}W(s,x(s)) \,ds \\
            & = W(t,x(t)) + \int_t^{\tau} \frac{\partial}{\partial t}W(s,x(s)) + \dot{x}(s)\cdot D_xW(s,x(s)) \,ds \\
            & = W(t,x(t)) + \int_t^{\tau} \frac{\partial}{\partial t}W(s,x(s)) + f(s,x(s),u(s))\cdot D_xW(s,x(s)) \,ds \\
            & \overset{\circledast}{\geq} W(t,x(t)) - \int_t^{\tau} L(s,x(s),u(s)) \,ds  
        \end{align*}

        Then:

        \[W(t,x(t)) \leq J(t,x;u)\]

        And therefore by taking the infimum over $\mathcal{U}$ and recalling $x(t)=x$ we get:

        \[W(t,x) \leq V(t,x)\]

        If furthermore $u^{\ast}$ satisfies \ref{1-2-uoptim} then the inequality $\overset{\circledast}{\geq}$ is an equality, and therefore:

        \[W(t,x) = J(t,x;u^{\ast})\]

        Which implies that $u^{\ast}$ is optimal and $W(t,x)=J(t,x;u^{\ast})=V(t,x)$. 
        The converse will be proved in a more general setting. In particular, only differentiability is needed.
        % Conversely, if $u^{\ast}$ is optimal:

        % \begin{align*}
        %     \color{red}{\int_t^{\tau} \frac{\partial}{\partial t}W(s,x^{\ast}(s)) + f(s,x^{\ast}(s),u^{\ast}(s)) \cdot D_xW(s,x^{\ast}(s)) \,ds & \geq - \int_t^{\tau^{\ast}} L(s,x^{\ast}(s),u^{\ast}(s)) \,ds  \\
        %     & \geq - \int_t^{\tau} L(s,x^{\ast}(s),u(s)) \,ds = }
        %     \color{black}{}
        % \end{align*}
    \end{proof} 
\end{theorem}


Theorem \ref*{1-2-Verificationthe} is an important tool in determining the explicit form of and optimal control. 
Indeed, condition \ref*{1-2-uoptim} can be restated as:

\begin{equation}\label{1-2-optimalityconditionu}
        u^{\ast}(s) \in \argmin_{v\in U} \left\{ f(s,x^{\ast}(s),v) \cdot D_xW(s,x^{\ast}(s)) + L(s,x^{\ast}(s),v)\right\}
\end{equation}

For almost all $s\in[t,t_1]$.

I have to prove verification theorem in more general case ($O\neq\R^n$).

We can express the optimality condition on $u$ \ref*{1-2-optimalityconditionu} in a differential inclusion form. 

\begin{corollary}
    A control $u^{\ast}$ is optimal if the corresponding state function $x^{\ast}$ satisfies:

    \begin{equation}
        x^{\ast} \in \left\{f(t,x,v) \,|\, v\in v^{\ast}(t,x)\right\}
    \end{equation}
\end{corollary}

% Necessary condition
\section{Pontryagin's principle and dynamic programming}

In the previous section we tackled the optimal control problem via dynamic programming.
As mentioned earlier this approach is of wide applicability and provides an implicit 
characterization of an optimal control. We now present another technique: the Pontryagin's principle.
As before, it will give rise to necessary condition on a control function 
to be optimal, but they'll come from a completely different perspective. We now present Pontryagin's 
principle in its full generality, and then we will see how it is connected with 
the dynamic programming approach. 

\subsection{Pontryagin's principle}

Pontryagin gives us an elegant and unintuitive way of solving problems of the kind:

\begin{equation}\label{1-3-problempon}
    \begin{cases}
        \dot{x}(s) = f(s,x(s),u(s)) & s\in [t,t_1] \\
        x(t) = x
    \end{cases}
\end{equation}

Where $u$ is bounded measurable into $U\subset\R^m$ and $O=\R^n$. Having defined the 
functional $J(t,x;u)$ as usual we define the \textit{control state Hamiltonian} 
as follows.

\begin{definition}
    The control state Hamiltonian of system \ref{1-3-problempon} is:

    \begin{equation}\label{1-3-hamiltonian}
        H(s,x,u,p) = - p \cdot f(s,x,u) - L(s,x,u)
    \end{equation}

    For all $s\in[t,t_1],\,x,p\in O,\,u\in U$.
\end{definition}

The variable $p$ is called \textit{costate} of the system. Pontryagin's principle 
gives us information about the costate under an optimal trajectory, which in turn will 
characterize the optimal control. 

\begin{theorem}\label{1-3-pontry}
    Let $u^{\ast}$ be an optimal control and $x^{\ast}$ its corresponding trajectory. 
    Then there exists a function $p^{\ast}:[t,t_1]\rightarrow O$ such that:

    \begin{equation}\label{1-3-pontry-x}
        \dot{x}^{\ast}(s) = D_p H(s,x^{\ast}(s),u^{\ast}(s),p^{\ast}(s)) 
    \end{equation}

    \begin{equation}\label{1-3-pontry-p}
        \dot{p}^{\ast}(s) =  - D_x H(s,x^{\ast}(s),u^{\ast}(s),p^{\ast}(s)) 
    \end{equation}

    And also:

    \begin{equation}\label{1-3-pontry-maxH}
        H(s,x^{\ast}(s),u^{\ast}(s),p^{\ast}(s)) = \sup_{v\in U} H(s,x^{\ast}(s),v,p^{\ast}(s))
    \end{equation}

    With:

    \begin{equation}\label{1-3-pontry-tras}
        p^{\ast}(t_1) = D \psi(x^{\ast}(t_1))
    \end{equation}
\end{theorem}

Thanks to this result we can determine an optimal control via the costate. By solving equation \ref{1-3-pontry-p} 
arisen in Theorem \ref{1-3-pontry} we can obtain the explicit form of $p^{\ast}$, and then 
retrieve $u^{\ast}$ from the maximization principle \ref{1-3-pontry-maxH}.


\subsection{Dynamic programming interplay}

As we just saw, Pontryagin's principle offers us an-even-though-quite-involved technique 
for finding an optimal control. As use of a control state Hamiltonian is crucial in this approach, it reminds 
us of the Hamiltonian defined in \ref{1-2-Hamiltonian1}. The similarity is also fortified by the maximization 
principle \ref{1-3-pontry-maxH}. We shall prove that this similarity unveils the direct link between 
Pontryagin's principle and the dynamic programming equation.

We will use the notion of differentiability of the value function. 
The classical notion of differentiability which we use is the following.

\begin{definition}
    V is differentiable in $(t,x)$ if there exists $V_t(t,x),V_x(t,x)\in\R$ such that:

    \begin{equation}
        \lim_{(h,k)\to(0,0)} \frac{1}{\abs{h}+\abs{k}}\abs{V(t+h,x+k)-V(t,x)-V_t(t,x)\cdot h-V_x(t,x)\cdot k}=0
    \end{equation}
\end{definition}

Differentiability is somewhat a strong hypothesis, but it allows us to prove the following
proposition. We must say that differentiability may easily fail in application, in such istances 
the notion of a weaker solution is needed, namely a viscosity solution.

\begin{theorem}
    Let $V$ be differentiable in $(t,x)\in Q$ and $u^{\ast}$ an optimal control
    such that $u^{\ast}\xrightarrow{s\to t}v$, then:

    \begin{equation}\label{1-3-dynamicprogrammeq}
        V_t(t,x) + L(t,x,v) + f(t,x,v)\cdot D_xV(t,x) = 0
    \end{equation}

    \begin{proof}
        Let $h>0$ s.t. $t+h<\tau$, then by \ref{1-2-propriformulation} we have:

        \[ V(t,x) = \int_t^{t+h} L(s,x(s),u^{\ast}(s)) \,ds + V(t+h,x(t+h))\]

        But because of differentiability we have:

        \[\lim_{h\to0}\frac{1}{\abs{h}}\abs{V(t+h,x(t+h))-V(t,x(t))} = V_t(t,x) + f(t,x,v)\cdot D_xV(t,x)\]

        Then we get:

        \begin{align}
        L(t,x,v) & = \lim_{h\to0} \frac{1}{\abs{h}} \int_t^{t+h} L(s,x(s),u(s)) \,ds = -\lim_{h\to0} \frac{1}{\abs{h}}\abs{V(t+h,x(t+h))-V(t,x(t))} \\
        & = - V_t(t,x) - f(t,x,v)\cdot D_xV(t,x)
        \end{align}
    \end{proof}
\end{theorem}

Furthermore, we impose existence and continuity of all derivatives of $f,L,g,\psi$.
The next theorem demonstrates that the costate in the Pontryagin Maximum
Principle is in fact the gradient in x of the value function v, taken along an optimal
trajectory. 

\begin{theorem}\label{1-3-pontrydyn}
    Let $u^{\ast}$ be an optimal right-continuous control and $x^{\ast}$ its corresponding trajectory. 
    Assume that the value function $V$ is differentiable at $(s,x^{\ast}(s))$ for $s\in[t,t_1)$. If we define:

    \begin{equation}\label{1-3-pontrydyn-defP}
        p(s) = D_x V(s,x^{\ast}(s)) 
    \end{equation}

    Then $p(s)$ satisfies \ref{1-3-pontry-p}, \ref{1-3-pontry-maxH} and \ref{1-3-pontry-tras}.

    \begin{proof}
        The trasversality condition \ref{1-3-pontry-tras} is straightforward from the definition of the  
        value function, which implies also the maximality condition \ref{1-3-pontry-maxH}. 
        We need to prove the "lagrangian multiplier condition" \ref{1-3-pontry-p}. 
        Let us drop all $^{\ast}$ and rewrite this differential equation:

        \begin{equation}\label{1-3-pontrydyn-rewritecond}
            \frac{d}{dt} p_j(s) = - \sum_{i=1}^n \frac{\partial}{\partial x_j}f_i(s,x(s),u(s))p_i(s) - \frac{\partial}{\partial x_j}L(s,x(s),u(s))
        \end{equation}

        This system admits solution $\overline{p}$ such that $\overline{p}(s)=D_xV(s,x(s))$; let us show that $\overline{p}(s)=p(s)$. Let $u_s$ be 
        the restriction of $u$ to $[r,t_1)$, which is admissible by assumption. We have:

        \[V(s,y) \leq J(s,y;u_s) \,\forall y\in\R^n\]

        Then, because $u$ is optimal, $y\mapsto J(s,y;u_s) - V(s,y)$ has its global minimum in $y=x(s)$, which implies by differentiability:

        \begin{equation}\label{1-3-pontrydyn-derconV}
            D_x V(s,x(s)) = D_x J(s,x(s);u_s)
        \end{equation}

        Then we prove that $\overline{p}(s) = D_x J(s,x(s);u_s)$. We denote $x(r)$ the solution starting at $x(s)$ at time $r$. Because $L\in C^1$ then for all $i=1,\dots,n$:

        \begin{equation}
            \frac{\partial}{\partial x_i}J(s,x(s),u) = \sum_{j=1}^n \int_s^{t_1} \left(L_{x_j}(r,x(r),u(r))\frac{\partial x_j(r)}{\partial x_i}\right) \,dr + \psi_{x_j}(x(t_1))\frac{\partial x_j(t_1)}{\partial x_i}
        \end{equation} 

        But then:

        \begin{align}
            \overline{p}_i(s)  & = \sum_{j=1}^n \frac{\partial x_j(s)}{\partial x_i}\overline{p}_j(s) = \sum_{j=1}^n \frac{\partial x_j(t_1)}{\partial x_i}\overline{p}_j(t_1) - \int_s^{t_1} \frac{d}{dr}\left(\sum_{j=1}^n \frac{\partial x_j(r)}{\partial x_i}\overline{p}_j(r)\right) \,dr \\
            & = \sum_{j=1}^n \frac{\partial x_j(t_1)}{\partial x_i}\psi(x(t_1)) - \sum_{j=1}^n \int_s^{t_1} \frac{d}{dr}\left(\frac{\partial x_j(r)}{\partial x_i}\right)\overline{p}_j(r) + \frac{\partial x_j(r)}{\partial x_i}\frac{d}{dr}\left(\overline{p}_j(r)\right) \,dr 
        \end{align}

        But under the integral:

        \[\int_s^{t_1} \frac{d}{dr}\sum_{l=1}^n \frac{\partial x_j(r)}{\partial x_i}\left(\frac{\partial}{\partial x_j}f_l(r,x(r),u(r))\overline{p}_j(r) - \sum_{l=1}^n \frac{\partial x_j(r)}{\partial x_i}\frac{\partial}{\partial x_j}f_l(r,x(r),u(r))\overline{p}_j(r) - \frac{\partial}{\partial x_j}L(r,x(r),u(r))\right) \,dr\]

        Then we get:

        \begin{align}
            \overline{p}_i(s) & = \sum_{j=1}^n \left(\frac{\partial x_j(t_1)}{\partial x_i}\psi(x(t_1)) + \int_s^{t_1} \frac{\partial}{\partial x_j}L(r,x(r),u(r))\frac{\partial x_j(r)}{\partial x_i} \,dr\right) \\
            & = \frac{\partial}{\partial x_i}J(s,x(s);u_r)
        \end{align}
    \end{proof}
\end{theorem}



% Sufficient Condition
\section{Existence}

We now prove an existence theorem for optimal controls.
We study the fixed time interval case with $O=\R^n$ and the function $f$ linear in $v$.
Furthermore, we impose convexity of $L$ in $v$. Under these assumptions a classical variational
argument proves the optimal control existence.
\begin{theorem}
    Let $U$ compact and convex, $f_1,f_2\in C^1(\overline{Q}\times U)$ such that $f(t,x,v)=f_1(t,x)+f_2(t,x)v$ and $\partial_xf_{1},\partial_xf_2,f_2$ bounded.
    Let also $L\in C^1(\overline{Q}\times U)$ and $L(t,x,\cdot)$ be convex for all $(t,x)\in\overline{Q}$ and the temrinal cost $\phi\in C(\R^n)$. Then there exist an optimal control $u^{\ast}(\cdot)$.

    \begin{proof}
        Let $u_n$ a minimizing sequence such that:

        \begin{equation}
            \lim_{n\to+\infty} J(t,x;u_n) = V(t,x)
        \end{equation}

        Let $x_n(\cdot)$ be the solutions to \ref{1-1-syst1} with $u=u_n$. If we show both sequence to converge respectively 
        (weakly) to $u^{\ast}$ and uniformly $x^{\ast}$ (along subsequences) such that the latter is again the solution to \ref{1-1-syst1} with $u=u^{\ast}$, then:

        \[J(t,x;u_n) = \int_t^{t_1} L(s,x^{\ast}(s),u_n(s)) \,ds + \int_t^{t_1} L(s,x_n(s),u_n(s)) - L(s,x^{\ast}(s),u_n(s)) \,ds + \phi(x_n(t_1))\]

        But then:

        \[\liminf_{n\to+\infty} \int_t^{t_1} L(s,x_n(s),u_n(s)) - L(s,x^{\ast}(s),u_n(s)) \,ds = 0\]

        And $\psi(x_n(t_1))\xrightarrow{n\to+\infty}\psi(x^{\ast}(t_1))$. But then:

        \begin{equation}\label{1-4-existenceliminf}
            \liminf_{n\to+\infty} J(t,x;u_n) = \liminf_{n\to+\infty} \int_t^{t_1} L(s,x^{\ast}(s),u_n(s)) \,ds \geq \int_t^{t_1} L(s,x^{\ast}(s),u^{\ast}(s)) \,ds 
        \end{equation}

        Because $L$ is convex in $u$\footnote{Explained in remark \ref{1-4-reamrkconvexity}}. Therefore:

        \[V(t,x)\leq J(t,x;u^{\ast}) \leq \liminf_{n\to+\infty} J(t,x;u_n) = V(t,x)\]

        We need to prove convergence of $x_n$ and $u_n$. Because $U$ is compact and \textit{convex} then $L^{\infty}([t,t_1];U)$ is 
        weakly sequentially compact. For what concerns $x_n$ we use Ascoli-Arzela's theorem to show that is admits a uniformly convergent subsequence. Being 
        uniformly limitated comes from:

        \begin{align}
            \abs{x_n(s)} & \leq \abs{x_n(s)-x}+\abs{x} = \abs{x} + \int_t^s \frac{d}{dr}\abs{x_n(r)} \,dr \\
            & = \abs{x} + \int_t^s \abs{f_1(r,x_n(r)) + f_2(r,x_n(r))\cdot u_n(r)}\,dr \\
            & \color{red}{\leq} \abs{x} + \int_t^s \norm{\partial_xf_1}_{\infty}\abs{x_n(r)} + \norm{f_2}_{\infty}\abs{u_n} \,dr \\
            & \leq C + K\left( \int_t^s \abs{x_n(r)} \,dr \right)   
        \end{align}

        Then by Gronwall's lemma $x_n$ is uniformly limitated. Equicontinuity comes from the uniform buondedness of the
        derivative $\dot{x}_n(s)$. Therefore, we know that there exist the weak limit $u^{\ast}$ and the uniform
        limit $x^{\ast}$. The latter is the solution of \ref{1-1-syst1} with $u=u^{\ast}$. Indeed:

        \begin{align*}
                x_n(s) & = x + \int_t^s\frac{d}{dr}x_n(r) \,dr\\
                & = x + \underbrace{\int_t^sf_1(r,x_n(r)) + f_2(r,x_n(r))u^{\ast}(r) \,dr}_{A_n} + \underbrace{\int_t^s \left[f_2(r,x_n(r)) - f_2(r,x^{\ast}(r))\right]\left[u_n(r)-u^{\ast}(r)\right] \,dr}_{B_n} \\
                & + \underbrace{\int_t^s f_2(r,x^{\ast}(r))\left[u_n(r)-u^{\ast}(r)\right] \,dr}_{C_n}
        \end{align*}
        
        Letting $n\to+\infty$ we get $B_n$ (by weak convergence and boundedness of $f_2$) and $C_n$ (by weak convergence) going to $0$ 
        while we obtain:

        \[A_n\xrightarrow{n\to+\infty}\int_t^sf_1(r,x^{\ast}(r))+f_2(r,x^{\ast}(r))u^{\ast}(r)\,dr\]
        
        And therefore the thesis.
    \end{proof}
\end{theorem}


\begin{remark}\label{1-4-reamrkconvexity}
    In the proof we asserted inequality \ref{1-4-existenceliminf} by convexity of the running cost. Indeed, 
    by convexity and being $C^1$:

    \[L(s,x(s),u_n(s)) \geq L(s,x(s),u^{\ast}(s)) + \left[u_n(s) - u^{\ast}(s)\right]L_u(s,x(s),u^{\ast}(s))\]

    Then by integrating and taking $\liminf_{n\to+\infty}$ we get the inequality (using weak convergence of $u_n$). 
\end{remark}

An existence result can also be proved in the context of $O\neq\R^n$, where the cost function
to be minimized has the form:

\begin{equation}
    J(t,x;u)=\int_t^{\tau} L(s,x(s),u(s)) \,ds + \Psi(\tau,x(\tau))
\end{equation}

with dynamical system described by $f(s,x,u)=u$. Therefore, the value function is:

\begin{equation}
    V(t,x) = \inf_{u\in\mathcal{U}} \left\{\int_t^{\tau} L(s,x(s),\dot{x}(s)) \,ds + \Psi(\tau,x(\tau))\right\}
\end{equation}

Under technical assumptions an optimal control exists, and the proof follows
the same procedure as before: we take a minimizing sequence, show its (weak) convergence
and its corresponding state evolution (uniform) convergence, then we show 
optimality, slightly modifying the proof because of convergence issues.

% \section{Linear Quadratic Regulator Problem}



\section{Infinite horizon}

A particularly interesting version of the maximization problem arising from optimal 
control is with infinite horizon. Let us consider the usual Cauchy' problem:

\begin{equation}
    \begin{cases}
        \dot{x}(s) = f(s,x(s),u(s)) & s\in[t,t_1] \\
        x(t) = x
    \end{cases}
\end{equation}

If we set $t_1=+\infty$ we get an infinite horizon problem. Thus, the maximization 
problem becomes:

\begin{equation}
    \inf_{u\in\mathcal{U}} \int_t^{\tau} L(s,x(s),u(s)) \, ds + g(\tau,x(\tau))\chi_{\tau<+\infty}
\end{equation}

Where $\tau$ is the exit time of $x(\cdot)$ from $\overline{O}$.\footnote{More precisely, is the exit time of $(s,x(s))$ from $\overline{Q}$.}

\section{Proof of Pontryagin's principle}

We will show the maximum Pontryagin's principle in the simple context of no running cost. 
A cleaver reconstruction of the non-zero running cost problem as a zero running cost one 
will enlarge the thesis to this situation. The first concept we will need is the variation 
of a control.

\begin{definition}
    Given $u\in\mathcal{U}$. For $\epsilon,r>0$ such that $0<r-\epsilon<r$ and $a\in U$ 
    we define the \textit{simple variation} $u_{\epsilon}\in\mathcal{U}$ such that:

    \begin{equation}\label{1-proofpontry-defvar}
        u_{\epsilon}(t) = \begin{cases}
            a & s\in(r-\epsilon,r) \\
            u(s) & s\notin(r-\epsilon,r)
        \end{cases}
    \end{equation}
\end{definition}

By defining the matrix $A:[0,+\infty)\rightarrow\R^{n\times n}: s\mapsto D_xf(s,x(s),u(s))$ we state 
the following lemma.

\begin{lemma}
    Let $x_{\epsilon}$ be solution of:

    \begin{equation}\label{1-proofpontry-dynamprob}
        \begin{cases}
            \dot{x}_{\epsilon}(s) = f(s,x_{\epsilon}(s),u_{\epsilon}(s)) & s\in[t,t_1] \\
            x_{\epsilon}(t) = x
        \end{cases}
    \end{equation}

    Then the solution is:

    \begin{equation}\label{1-proofpontry-solvar}
        x_{\epsilon}(s) = x(s) + \epsilon y(s) + o(\epsilon)\, \epsilon\to0
    \end{equation}

    Where $y\equiv0$ on $[t,r]$ and:

    \begin{equation}
        \begin{cases}
            \dot{y}(s) = A(s)y(s) & s\in[r,t_1]\\
            y(r) = y^r
        \end{cases}
    \end{equation}

    With $y^r=f(x(r),a)-f(x(r),u(r))$.

    \begin{proof}
        Let us divide the proof in three cases.

        \begin{itemize}
            \item $s\in[t,r-\epsilon]$: then $y(t)=0$ and $u_{\epsilon}(t)=u(t)$, therefore:
            
            \[x_{\epsilon}(t)=x(t)=x(t)+\epsilon y(t)+o(\epsilon)\]

            \item $s\in(r-\epsilon,r)$: then we have:
            
            \[x_{\epsilon}(s)-x(s) = \int_{r-\epsilon}^s f(w,x_{\epsilon}(w),u_{\epsilon}(w))-f(w,x(w),u(w)) \,dw + o(\epsilon)\]

            Which is a little \textit{o} of $\epsilon$ (because $f$ is continuous). 

            \item $s\in[r,t_1]$: from before if $s=r$ then:
            
            \begin{align}
                x_{\epsilon}(r)-x(r) & = \int_{r-\epsilon}^r f(w,x_{\epsilon}(w),u_{\epsilon}(w))-f(w,x(w),u(w)) \,dw + o(\epsilon) \\
                & = \lim_{w\to s}[f(w,x_{\epsilon}(w),u_{\epsilon}(w))-f(w,x(w),u(w))]\epsilon+ o(\epsilon) \\
                & = y^s\epsilon + o(\epsilon)
            \end{align}

            \color{red}{If $s>r$ then:}


        \end{itemize}
    \end{proof}
\end{lemma}

Let us now prove Pontryagin's principle with no running cost. The payoff functional is:

\begin{equation}\label{1-proofpontry-norunfunct}
    J(t,x;u) = \psi(x(t_1))
\end{equation}

And therefore the Hamiltonian is:

\begin{equation}\label{1-proofpontry-norunham}
    H(s,x,u,p) = - f(s,x,u)\cdot p 
\end{equation}

\begin{theorem}\label{1-proofpontry-theo}
    There exists a function $p^{\ast}:[t,t_1]\rightarrow \R^n$ such that:

    \begin{equation}
        \dot{p}^{\ast}(s) = -D_x H(s,x^{\ast}(s),u^{\ast}(s),p^{\ast}(s))\, s\in[t,t_1]
    \end{equation}

    together with the maximization:

    \begin{equation}
        H(s,x^{\ast}(s),u^{\ast}(s),p^{\ast}(s)) = \sup_{v\in U} H(s,x^{\ast}(s),v,p^{\ast}(s))
    \end{equation}

    and the trasversality condition:

    \begin{equation}
        p^{\ast}(t_1) = D\psi(x(t_1))
    \end{equation}

    \begin{proof}
        Let us drop all the $^{\ast}$. Let $p$ be the unique solution of:

        \begin{equation}
            \begin{cases}
                \dot{p}(s) = - A'(s)\cdot p(s) & s\in[t,t_1] \\
                p(t_1) = D\psi(x(t_1))
            \end{cases}
        \end{equation}

        It exists and is unique because the latter is a linear differential equation with integrable coefficient. 
        We already satisfy the trasversality condition and the adjoint dynamics. We prove the 
        maximization principle. Let $a\in U$. We define the variation $u_{\epsilon}$ for $\epsilon,r\in(t,t_1)$ as before.
        Since $\epsilon\mapsto  J(t,x;u_{\epsilon})$ for $\epsilon\in[0,1]$ has a maximum in $\epsilon=0$ we have:
        
        \begin{equation}\label{1-proofpontry-dervar0}
            \frac{d}{d\epsilon}J(t,x;u_{\epsilon})\leq0
        \end{equation}

        Computing the derivative, using \ref{1-proofpontry-defvar}:

        \begin{align}
            \frac{d}{d\epsilon}J(t,x;u_{\epsilon})\big|_{\epsilon=0} & = \frac{d}{d\epsilon}\psi(x_{\epsilon}(t_1))\big|_{\epsilon=0} \\
            & = \frac{d}{d\epsilon}\psi(x(t_1) + \epsilon y(t_1) + o(\epsilon)) = D\psi(x(t_1))\cdot y(t_1) \\
            & = p(t_1)\cdot y(t_1) = p(r)\cdot[f(r,x(r),a)-f(r,x(r),u(r))]
        \end{align}

        Where the last equality comes from:

        \begin{align*}
            \frac{d}{ds}\left(p(s)\cdot y(s)\right) & = \dot{p}(s)\cdot y(s) + p(s)\cdot \dot{y}(s) \\
            & = -A'(s)\cdot p(s)\cdot y(s) + p(s) \cdot A(s)\cdot y(s) = 0
        \end{align*}

        Therefore, by plugging into \ref{1-proofpontrydervar0} we get:

        \[0\geq p(r)\cdot[f(r,x(r),a)-f(r,x(r),u(r))]\]

        Which implies:

        \[H(r,x(r),a,p(r)) = f(r,x(r),a) \cdot p(r) \leq f(r,x(r),u(r)) \cdot p(r) = H(r,x(r),u(r),p(r))\]
    \end{proof}
\end{theorem}

Given Pontryagin's principle for no running cost problems we can extend the result to the general case:

\begin{equation}
    J(t,x;u) = \int_t^{t_1} L(s,x(s),u(s)) \,ds + \psi(x(t_1))
\end{equation}

Where the Hamiltonian is:

\begin{equation}
    H(s,x,u,p) = f(s,x,u)\cdot p + L(s,x,u)
\end{equation}

Indeed, theorem \ref{1-proofpontry-theo} holds also under these conditions. We rewrite the problem as it has no 
running cost and then apply the theorem. Let us define $x^{n+1}$ as:

% \begin{equation}
%     \begin{cases}
%         \dot{x}^{n+1}(s) = L(s,x^{n+1}(s),u(s)) & s\in [t,t_1] \\
%         x^{n+1}(0) = 0 
%     \end{cases}
% \end{equation}

\begin{equation}
    x^{n+1}(s) = \int_t^s L(w,x(w),u(w)) \,dw
\end{equation}

Then by defining $\overline{f}, \overline{g}, \overline{x},\overline{x}(s)$ as:

\begin{equation}
    \overline{f}(s,x,u)=\begin{bmatrix}
        f(s,x,u) \\
        L(s,x,u)
    \end{bmatrix},\,\overline{x}=\begin{bmatrix}
        x \\
        0
    \end{bmatrix},\,\overline{x}(s)=\begin{bmatrix}
        x(s) \\
        x^{n+1}(s)
    \end{bmatrix},\,\overline{g}(\overline{x}(t_1)) = g(x(t_1)) + x^{n+1}(t_1)
\end{equation}

Thus, the problem has no running cost. We can apply the theorem and, noticing that 
$p^{n+1}\equiv1$ we get the thesis.
% \section{Pontryagin} %Maybe in dynamic programming?

% \section{Dynamic programming}

% \section{Existence theorem}

\end{document}