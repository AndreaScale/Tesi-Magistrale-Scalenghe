\section{Dynamic programming principle}

One way of tackling certain optimal control problems is via \textit{dynamic programming}. 
% The idea is to use a \textit{value function} as a tool to solve the control problem, finding sufficient and sometimes necessary conditions.
Let us define the \textit{value function}:

\begin{equation}\label{1-2-valuefunc}
    V(t,x)=\inf_{u\in \mathcal{U}(t,x)}J(t,x;u)
\end{equation}

For all $(t,x)\in\overline{Q}$. We get rid of the instance in which $V(t,x)=-\infty$ assuming $Q$ to be compact, or $L$ and $\Psi$ to be bounded below.
We aim at retrieving the argument which attains the infimum of \ref*{1-2-valuefunc}. In order to immerse this optimal control problem into 
a dynamic programming one we see the state of the system as the state of the variable and the control function as the decision function.
The basic idea behind dynamic programming techniques is to subdivide a problem into smaller problems, what does this mean in our context?
We will be able to find instantaneous the value function $V$ via a partial differential equation (PDE) called Hamilton-Jacobi-Bellman equation.

We start by stating and proving the following proposition, which provides us with an equivalent definition of the value function.

\begin{proposition}\label{1-2-propriformulation}
    For any $(t,x)\in\overline{Q}$ and any $r\in I$ then:

    \begin{equation}\label{1-2-riformulationvalue}
        V(t,x)=\inf_{u\in\mathcal{U}(t,x)}\left\{\int_t^{r\land\tau}L(s,x(s),u(s))\,ds+g(\tau,x(\tau))\chi_{\tau<r}+V(r,x(r))\chi_{r\leq\tau}\right\}
    \end{equation}

    \begin{proof}
        Value function less than rhs. If $r>\tau$ then $\tau<t_1$ and $\Psi(r\land\tau,x(r\land\tau))=g(\tau,x(\tau))$ and then \ref*{1-2-riformulationvalue}
        follows directly by definition. If $r\leq\tau$, let $\delta>0$ then there exists $u^1\in\mathcal{U}(r,x(r))$ such that:

        \[\int_r^{\tau^1}L(s,x^1(s),u^1(s))\,ds+\Psi(\tau^1,x^1(\tau^1))\leq V(r,x(r))+\delta\]

        Where $x^1$ is the state function corresponding to $u^1$ with initial condition $(r,x(r))$ and $\tau^1$ the first exit from $\overline{Q}$ of $(s,x^1(s))$.  
        By defining $\tilde{u}$ as for the switching condition \ref{1-1-switcond} we have $\tau^1=\tilde{\tau}$, because $\tau\geq r$ and then $\tilde{u}$ is $u^1$. Then:
        
        \begin{align*}
            V(t,x) & \leq V(t,x;\tilde{u}) \\
            & = \int_t^{\tilde{\tau}} L(s,\tilde{x}(s),\tilde{u}(s))\,ds+\Psi(\tilde{\tau},\tilde{x}(\tilde{\tau})) \\
            & = \int_t^{r} L(s,x(s),u(s))\,ds+\int_r^{\tau^1} L(s,x^1(s),u^1(s))\,ds+\Psi(\tau^1,x^1(\tau^1)) \\
            & \leq \int_t^{r} L(s,x(s),u(s))\,ds + V(r,x(r)) + \delta
        \end{align*}

    Since $\delta$ is arbitrary the first inequality is proved.
    
    \noindent Value function is bigger than rhs. Let $\delta>0$ and $U\in\mathcal{U}(t,x)$ such that:

    \[\int_r^{\tau}L(s,x(s),u(s))\,ds+\Psi(\tau,x(\tau))\leq V(t,x)+\delta\]

    Then: 

    \begin{align*}
        V(t,x) & \geq \int_r^{\tau}L(s,x(s),u(s))\,ds+\Psi(\tau,x(\tau)) - \delta \\
        % & = \int_t^{\tilde{\tau}} L(s,\tilde{x}(s),\tilde{u}(s))\,ds+\Psi(\tilde{\tau},\tilde{x}(\tilde{\tau})) \\
        & = \int_t^{r\land\tau} L(s,x(s),u(s))\,ds+\int_{r\land\tau}^{\tau} L(s,x(s),u(s))\,ds+\Psi(\tau,x(\tau)) -\delta\\
        % & \leq \int_t^{r} L(s,x(s),u(s))\,ds + V(r,x(r)) + \delta
        & = \int_t^{r\land\tau} L(s,x(s),u(s))\,ds+J(r,x(r))\chi_{r\leq\tau}+g(\tau,x(\tau))\chi_{\tau<r} - \delta\\
        & = \int_t^{r\land\tau} L(s,x(s),u(s))\,ds+V(r,x(r))\chi_{r\leq\tau}+g(\tau,x(\tau))\chi_{\tau<r} - \delta
    \end{align*}

    As $\delta$ is arbitrary we proved the proposition.
    \end{proof}
\end{proposition}


In the proof we used the concept of $\delta$-\textit{optimal control}, that is the control function $u\in\mathcal{U}(r,x(r))$ such that:

\[\int_r^{\tau^1}L(s,x^1(s),u^1(s))\,ds+\Psi(\tau^1,x^1(\tau^1))\leq V(r,x(r))+\delta.\]

This new representation allows us to find the so-called \textit{dynamic programming equation}. We have to impose that the value function 
is continuously differentiable, although this is not always the case. If differentiability fails, the notion of viscosity solution is needed. 

We restrict our study to $O=\R^n$, as the other cases need viscosity solutions. In this situation the boundary condition is simply:

\begin{equation}\label{1-2-boundcondR}
    V(t_1,x)=\psi(x)\,\forall x\in\R^n
\end{equation}

Before stating the fundamental theorem which gives sufficient conditions 
for a solution to the optimal problem we follow a heuristic reasoning which will help our intuition. Under the hypothesis of continuous differentiability of the value function 
let us rewrite the dynamic programming principle as:

\begin{equation}
    \inf_{u\in\mathcal{U}}\left\{\frac{1}{h}\int_t^{(t+h)\land\tau} L(s,x(s),u(s))\,ds + \frac{1}{h}g(\tau,x(\tau))\chi_{\tau<t+h} + \frac{1}{h}\left[V(t+h,x(t+h))\chi_{\tau\geq t+h} - V(t,x)\right]\right\}=0
\end{equation}

Then if we formally let $h\to0$ we get:

\[\inf_{u\in\mathcal{U}}\left\{L(t,x(t),u(t)) + \partial_tV(t,x(t)) + D_xV(t,x(t))\cdot f(t,x(t),u(t))\right\}=0\]

Which can be rewritten as:

\begin{equation}\label{1-2-HJB1}
    -\frac{\partial}{\partial t}V(t,x) + H(t,x,D_xV(t,x))=0
\end{equation}

Where for $(t,x,p)\in \overline{Q}\times\R^n$ the Hamiltonian is defined as:

\begin{equation}\label{1-2-Hamiltonian1}
    H(t,x,p) = \sup_{v\in\R^n}\left\{-p\cdot f(t,x,v) - L(t,x,v)\right\}.
\end{equation}

Equation \ref{1-2-HJB1} turns out to be the main sufficient condition for the control to be optimal.

\begin{theorem}[Verification Theorem]\label{1-2-Verificationthe}
    Let $W\in C^1(\overline{Q})$ satisfy \ref{1-2-HJB1} and \ref{1-2-boundcondR} then:

    \[W(t,x)\leq V(t,x)\]

    Moreover, if there exists $u^{\ast}\in\mathcal{U}$ such that:

    \[L(s,x^{\ast}(s),u^{\ast}(s)) + f(s,x^{\ast},u^{\ast}(s))\cdot D_xW(s,x^{\ast}(s)) = - H(s,x^{\ast}(s),D_xW(s,x^{\ast}(s)))\]

    For almost all $s\in[t,t_1]$, then $u^{ast}$ is optimal and $W=V$.

    \begin{proof}
        
    \end{proof}
\end{theorem}