\documentclass[12pt,a4paper]{book}
% \usepackage{times}
\usepackage[a4paper, margin=1.25in]{geometry}
\usepackage[utf8]{inputenc}
\usepackage[english]{babel}
\selectlanguage{english}
\usepackage{amsmath,amsthm,amssymb,fge,mathrsfs}
\usepackage{bookmark}
\usepackage{mathtools}
% \usepackage{hyperref}
\usepackage{hyperref}

\usepackage{csquotes}
\usepackage{makeidx} %per fare l'indice
\usepackage{faktor} %per i quozienti

%tikz (pacchetto per i disegni)
\usepackage{tikz}
\usetikzlibrary{arrows}
\usetikzlibrary{tikzmark}
\usetikzlibrary{calc} %per poter fare i calcoli
\usetikzlibrary{arrows.meta} %per usare veri tipi di frecce
\usetikzlibrary{calc,patterns,angles,quotes} %per disegnare gli angoli
\usetikzlibrary{decorations} %per i grafici orientati (con sopra le frecce)
\usetikzlibrary{decorations.markings}
\usetikzlibrary{backgrounds} %per il colore sullo sfondo delle immagini
\usetikzlibrary{shapes.geometric} %per i triangoli negli alberi

\usepackage{pgfplots}
\pgfplotsset{compat=newest}
\usepgfplotslibrary{fillbetween} %per colorare le aree comprese tra due grafici

\usepackage{caption} %per la caption sotto alle immagini
\usepackage[tracking=true]{microtype} %per avvicinare il testo
\usepackage{enumitem} %per modificare agilmente gli elenchi
\usepackage{esvect} %per le freccette di vettore sui caratteri
\usepackage{stmaryrd} %fulmine per il simbolo dell'assurdo
\usepackage{relsize} %per fare i simboli più grandi
\usepackage{mathabx} %per il simbolo di dotminus
\usepackage{esint} %per l'integrale tagliato (fint)

\usepackage[customcolors]{hf-tikz} %per evidenziare pezzi di matrice

\usepackage{verbatim} %per commentare un blocco

\usepackage{multirow} %per la tabella
\usepackage{array} %per lo spazio nella tabella

%Libreria per gli alberi
\usepackage[linguistics]{forest}

%Librerie aggiunte
\usepackage{multicol} %per gli elenchi puntati su più colonne/righe
\usepackage{multirow}

%pacchetto per il prodotto interno
\usepackage{physics}

%Libreria per i subfiles

\usepackage{subfiles} %meglio caricarla per ultima

\usepackage{biblatex}
% \addbibresource{Bibliography.bib}
\addbibresource{Bibliography.bib}


% \addbibresource{biblio.bib}
%usiamo paragraph come nuova sottosottosottosezione 
\makeatletter
\renewcommand\paragraph{\@startsection{paragraph}{4}{\z@}%
            {-2.5ex\@plus-1ex \@minus -.25ex}%
            {1.25ex \@plus .25ex}%
            {\normalfont\normalsize\bfseries}}
\makeatother
\setcounter{secnumdepth}{4} % how many sectioning levels to assign numbers to
\setcounter{tocdepth}{4}    % how many sectioning levels to show in ToC

\newtheorem{theorem}{Theorem}
\numberwithin{theorem}{section}
\newtheorem*{theorem*}{Theorem}
\newtheorem{corollary}[theorem]{Corollary}
\newtheorem{proposition}[theorem]{Proposition}
\newtheorem{lemma}[theorem]{Lemma}
\newtheorem*{lemma*}{Lemma}

\newtheorem{definition}{Definition}
\numberwithin{definition}{section}

\newtheorem*{remark}{Remark}

\newtheorem{example}{Example}
\numberwithin{example}{section}

\newtheorem{exercise}{Exercise}
\numberwithin{exercise}{section}

\renewcommand{\proofname}{\proof}

\renewcommand{\thefootnote}{[\arabic{footnote}]}% Modify footnote globally

%%%simboli speciali
\newcommand{\N}{\mathbb{N}}
\newcommand{\Z}{\mathbb{Z}}
\newcommand{\Q}{\mathbb{Q}}
\newcommand{\R}{\mathbb{R}}
\newcommand{\C}{\mathbb{C}}

\newcommand{\CO}{\mathcal{C}}

\DeclareMathOperator{\Dom}{Dom}
\DeclareMathOperator{\Ima}{Im}

\DeclareMathOperator{\mcd}{MCD}
\DeclareMathOperator{\mcm}{mcm}
\DeclareMathOperator{\supp}{supp}
\DeclareMathOperator{\spanlin}{span}

%\DeclarePairedDelimiter\abs{\lvert}{\rvert}
%\DeclarePairedDelimiter\norm{\lVert}{\rVert}

%Necessario per far si che la dimensione del valore assoluto e della norma si adatti all'aromento passato
%\makeatletter
%\let\oldabs\abs
%\def\abs{\@ifstar{\oldabs}{\oldabs*}}
%
%\let\oldnorm\norm
%\def\norm{\@ifstar{\oldnorm}{\oldnorm*}}
%\makeatother

%rinomino il comando per il prodotto interno
\let\pint\braket
%comando per il prodotto scalare
\newcommand{\pscal}[2]{\left\langle #1, #2 \right\rangle}

%Spaziatura per i quantificatori
\let\oldforall\forall
\renewcommand{\forall}{\; \oldforall \;}
\let\oldexists\exists
\renewcommand{\exists}{\; \oldexists \;}

%freccia per la convergenza debole
\newcommand{\longrightharpoonup}{\xrightharpoonup{\phantom{AB}}}

\newcommand{\warrow}{\xrightharpoonup{\:\, w \:\, }}
\newcommand{\sarrow}{\overset{s}{\longrightarrow}}
\newcommand{\wastarrow}{\xrightharpoonup{ w^* }}


\setlength{\parindent}{0in} 
\title{Title}
\author{Andrea Scalenghe}
\date{March 2024}



\begin{document}
\thispagestyle{empty}

\centerline {\Large{\textsc{ UNIVERSIT\`A DEGLI STUDI DI TORINO}}}
\vskip 20 pt

\centerline {\Large{\textsc DIPARTIMENTO DI MATEMATICA GIUSEPPE PEANO}}

\vskip 20 pt

\centerline {{\textsc SCUOLA DI SCIENZE DELLA NATURA}}

\vskip 20 pt

\centerline {\Large{\textsc Corso di Laurea Magistrale in Matematica}}
\vskip 60 pt





%\begin{tabular}{ccc}
\centerline {\includegraphics[width=4cm]{logo_new_2022.png}}
%   \end{tabular}

\vskip 1.2cm

\centerline {\normalsize {Tesi di Laurea  Magistrale}} 

\vskip 0.7cm

\centerline {\Large {\bf TITOLO}}

\vskip 1.7cm

\noindent Relatrice: Susanna Terracini
\hfill  {Candidato: Andrea Scalenghe}

\noindent Co-relatrice: Elena Issoglio


\vskip 2.7cm


\centerline{ANNO ACCADEMICO 2023-2024}


\newpage

{\hypersetup{linkbordercolor=black, hidelinks}
% or \hypersetup{linkcolor=black}, if the colorlinks=true option of hyperref is used
\tableofcontents
}
\newpage

\hypersetup{colorlinks, linkcolor = gray, citecolor = gray, hidelinks}

\chapter{Introduction}

In this dissertation I will adrress optimal control theory both from a deterministic and stochastic point of view. 

\newpage

\chapter{Deterministic Optimal Control}

We now study deterministic optimal control. The system we aim to control is governed by differential equations. 

Short description of what is done in this chapter.

\section{Finite horizon}


Let us consider a finite interval $I=[t,t_1]\subset\R$ as the operating time of a system. At each time $s\in I$, the system is described by $x(s)\in O\subseteq \R^n$ and controlled by $u(s)\in U\subseteq\R^n$
where $U$ is called control space. The system's evolution is described by

\begin{equation}\label{1-1-syst1}
    \begin{cases}
        \dot{x}(s) = f(s,x(s),u(s)) & s\in I \\
        x(t) = x,
    \end{cases}
\end{equation}

for given $x\in O$ and suitable $f:\overline{Q}\times U \rightarrow \R^m$, where $Q=[t,t_1)\times O$. We impose $f\in C(\overline{Q}\times U)$ and the existence of $K_{\rho}>0$ such that for all $\rho>0$ then

\begin{equation}\label{1-1-lipsch}
    \abs{f(t,x,v)-f(t,y,v)} \leq K_{\rho}\abs{x-y},
\end{equation}

for all $t\in I$, $x,y\in O$ and $v\in U$ such that $\abs{v}\leq\rho$. Under this conditions the system \eqref{1-1-syst1} has a unique solution. 
Controls $u(\cdot)$ are assumed to be in the set $L^{\infty}\left([t,t_1];U\right)$. We will soon specify more about the set of controls.

We have described a control problem. The concept of optimality is related to a value function, specified by payoffs (or costs) associated to the system's states.
Let $L\in C(\overline{Q}\times U)$ be the \textit{running cost}, and $\Psi\in C(I\times O)$ the \textit{terminal cost} defined as

\begin{equation}\label{1-1-deftermcost}
    \Psi(t,x) = \begin{cases}
        g(t,x) & \text{if } (t,x)\in [t,t_1)\times O \\
        \psi(x) & \text{if } (t,x)\in \{t_1\}\times O.  
    \end{cases}
\end{equation}

We define the \textit{payoff} $J$ as

\begin{equation}\label{1-1-payoff1}
    J(t,x;u) = \int_t^{\tau}L(s,x(s),u(s)) \,ds + \Psi(\tau, x(\tau)),
\end{equation}

where $\tau$ is the exit time of $(s,x(s))$ from $\overline{Q}$, that is

\begin{equation}\label{1-1-taudef}
    \tau = \begin{cases}
        \inf\{s\in [t,t_1)\,|\, x(s)\notin \overline{O}\} & \text{if } \exists s\in [t,t_1): x(s)\notin\overline{O} \\
        t_1 & \text{if }  x(s)\in\overline{O}\,\forall s\in [t,t_1).
    \end{cases}
\end{equation}

A control $u^{\ast}(\cdot)$ is \textit{optimal} if

\begin{equation}\label{1-1-optimalcondition}
    J(t,x;u^{\ast}) \leq J(t,x;u) \quad \forall u\in L^{\infty}(I;U).
\end{equation}

We have to impose a further condition on controls, the \textit{switching condition}. Heuristically, we ask controls to be consistent as time passes, that is we can switch from one control to another at a given point in time without losing control's property. 
Let us assume that we have $u\in\mathcal{U}(t,x)$ and $u'\in\mathcal{U}(r,x(r))$ for $r\in[t,\tau]$. If we define

\begin{equation}\label{1-1-switcond}
    \tilde{u}(s)=\begin{cases}
        u(s) & s\in[t,r) \\
        u'(s) & s\in[r,t_1],
    \end{cases}
\end{equation}

then we impose

\begin{equation}
    \tilde{u}_s\in\mathcal{U}(s,\tilde{x}(s)) \quad \forall s\in[t,\tilde{\tau}],
\end{equation}

where $\tilde{x}$ is the solution to the control problem \eqref{1-1-syst1} with control $\tilde{u}$ and initial condition $x$, 
$\tilde{u}_s$ is the restriction of $\tilde{u}$ to $[s,t_1]$, and $\tilde{\tau}$ is the exit time of $(s,\tilde{x}(s))$ from $\overline{Q}$.
This condition assures that admissible controls can be replaced as the time evolves and the resulting control is still admissible. 

\section{Dynamic programming principle}

One way of tackling certain optimal control problems is via \textit{dynamic programming}. 
% The idea is to use a \textit{value function} as a tool to solve the control problem, finding sufficient and sometimes necessary conditions.
Let us define the \textit{value function}:

\begin{equation}\label{1-2-valuefunc}
    V(t,x)=\inf_{u\in \mathcal{U}(t,x)}J(t,x;u)
\end{equation}

For all $(t,x)\in\overline{Q}$. We get rid of the instance in which $V(t,x)=-\infty$ assuming $Q$ to be compact, or $L$ and $\Psi$ to be bounded below.
We aim at retrieving the argument which attains the infimum of \ref*{1-2-valuefunc}. In order to immerse this optimal control problem into 
a dynamic programming one we see the state of the system as the state of the variable and the control function as the decision function.
The basic idea behind dynamic programming techniques is to subdivide a problem into smaller problems, what does this mean in our context?
We will be able to find instantaneous the value function $V$ via a partial differential equation (PDE) called Hamilton-Jacobi-Bellman equation.

We start by stating and proving the following proposition, which provides us with an equivalent definition of the value function.

\begin{proposition}\label{1-2-propriformulation}
    For any $(t,x)\in\overline{Q}$ and any $r\in I$ then:

    \begin{equation}\label{1-2-riformulationvalue}
        V(t,x)=\inf_{u\in\mathcal{U}(t,x)}\left\{\int_t^{r\land\tau}L(s,x(s),u(s))\,ds+g(\tau,x(\tau))\chi_{\tau<r}+V(r,x(r))\chi_{r\leq\tau}\right\}
    \end{equation}

    \begin{proof}
        Value function less than rhs. If $r>\tau$ then $\tau<t_1$ and $\Psi(r\land\tau,x(r\land\tau))=g(\tau,x(\tau))$ and then \ref*{1-2-riformulationvalue}
        follows directly by definition. If $r\leq\tau$, let $\delta>0$ then there exists $u^1\in\mathcal{U}(r,x(r))$ such that:

        \[\int_r^{\tau^1}L(s,x^1(s),u^1(s))\,ds+\Psi(\tau^1,x^1(\tau^1))\leq V(r,x(r))+\delta\]

        Where $x^1$ is the state function corresponding to $u^1$ with initial condition $(r,x(r))$ and $\tau^1$ the first exit from $\overline{Q}$ of $(s,x^1(s))$.  
        By defining $\tilde{u}$ as for the switching condition \ref{1-1-switcond} we have $\tau^1=\tilde{\tau}$, because $\tau\geq r$ and then $\tilde{u}$ is $u^1$. Then:
        
        \begin{align*}
            V(t,x) & \leq V(t,x;\tilde{u}) \\
            & = \int_t^{\tilde{\tau}} L(s,\tilde{x}(s),\tilde{u}(s))\,ds+\Psi(\tilde{\tau},\tilde{x}(\tilde{\tau})) \\
            & = \int_t^{r} L(s,x(s),u(s))\,ds+\int_r^{\tau^1} L(s,x^1(s),u^1(s))\,ds+\Psi(\tau^1,x^1(\tau^1)) \\
            & \leq \int_t^{r} L(s,x(s),u(s))\,ds + V(r,x(r)) + \delta
        \end{align*}

    Since $\delta$ is arbitrary the first inequality is proved.
    
    \noindent Value function is bigger than rhs. Let $\delta>0$ and $U\in\mathcal{U}(t,x)$ such that:

    \[\int_r^{\tau}L(s,x(s),u(s))\,ds+\Psi(\tau,x(\tau))\leq V(t,x)+\delta\]

    Then: 

    \begin{align*}
        V(t,x) & \geq \int_r^{\tau}L(s,x(s),u(s))\,ds+\Psi(\tau,x(\tau)) - \delta \\
        % & = \int_t^{\tilde{\tau}} L(s,\tilde{x}(s),\tilde{u}(s))\,ds+\Psi(\tilde{\tau},\tilde{x}(\tilde{\tau})) \\
        & = \int_t^{r\land\tau} L(s,x(s),u(s))\,ds+\int_{r\land\tau}^{\tau} L(s,x(s),u(s))\,ds+\Psi(\tau,x(\tau)) -\delta\\
        % & \leq \int_t^{r} L(s,x(s),u(s))\,ds + V(r,x(r)) + \delta
        & = \int_t^{r\land\tau} L(s,x(s),u(s))\,ds+J(r,x(r))\chi_{r\leq\tau}+g(\tau,x(\tau))\chi_{\tau<r} - \delta\\
        & = \int_t^{r\land\tau} L(s,x(s),u(s))\,ds+V(r,x(r))\chi_{r\leq\tau}+g(\tau,x(\tau))\chi_{\tau<r} - \delta
    \end{align*}

    As $\delta$ is arbitrary we proved the proposition.
    \end{proof}
\end{proposition}


In the proof we used the concept of $\delta$-\textit{optimal control}, that is the control function $u\in\mathcal{U}(r,x(r))$ such that:

\[\int_r^{\tau^1}L(s,x^1(s),u^1(s))\,ds+\Psi(\tau^1,x^1(\tau^1))\leq V(r,x(r))+\delta.\]

This new representation allows us to find the so-called \textit{dynamic programming equation}. We have to impose that the value function 
is continuously differentiable, although this is not always the case. If differentiability fails, the notion of viscosity solution is needed. 

We restrict our study to $O=\R^n$, as the other cases need viscosity solutions. In this situation the boundary condition is simply:

\begin{equation}\label{1-2-boundcondR}
    V(t_1,x)=\psi(x)\,\forall x\in\R^n
\end{equation}

Before stating the fundamental theorem which gives sufficient conditions 
for a solution to the optimal problem we follow a heuristic reasoning which will help our intuition. Under the hypothesis of continuous differentiability of the value function 
let us rewrite the dynamic programming principle as:

\begin{equation}
    \inf_{u\in\mathcal{U}}\left\{\frac{1}{h}\int_t^{(t+h)\land\tau} L(s,x(s),u(s))\,ds + \frac{1}{h}g(\tau,x(\tau))\chi_{\tau<t+h} + \frac{1}{h}\left[V(t+h,x(t+h))\chi_{\tau\geq t+h} - V(t,x)\right]\right\}=0
\end{equation}

Then if we formally let $h\to0$ we get:

\[\inf_{u\in\mathcal{U}}\left\{L(t,x(t),u(t)) + \partial_tV(t,x(t)) + D_xV(t,x(t))\cdot f(t,x(t),u(t))\right\}=0\]

Which can be rewritten as:

\begin{equation}\label{1-2-HJB1}
    -\frac{\partial}{\partial t}V(t,x) + H(t,x,D_xV(t,x))=0
\end{equation}

Where for $(t,x,p)\in \overline{Q}\times\R^n$ the Hamiltonian is defined as:

\begin{equation}\label{1-2-Hamiltonian1}
    H(t,x,p) = \sup_{v\in\R^n}\left\{-p\cdot f(t,x,v) - L(t,x,v)\right\}.
\end{equation}

Equation \ref{1-2-HJB1} turns out to be the main sufficient condition for the control to be optimal.

\begin{theorem}[Verification Theorem]\label{1-2-Verificationthe}
    Let $W\in C^1(\overline{Q})$ satisfy \ref{1-2-HJB1} and \ref{1-2-boundcondR} then:

    \[W(t,x)\leq V(t,x)\]

    Moreover, there exists $u^{\ast}\in\mathcal{U}$ such that:

    \begin{equation}\label{1-2-uoptim}
        L(s,x^{\ast}(s),u^{\ast}(s)) + f(s,x^{\ast},u^{\ast}(s))\cdot D_xW(s,x^{\ast}(s)) = - H(s,x^{\ast}(s),D_xW(s,x^{\ast}(s)))
    \end{equation}

    For almost all $s\in[t,t_1]$ if and only if $u^{\ast}$ is optimal and $W=V$.

    \begin{proof}
        Let $u\in\mathcal{U}$, then:

        \begin{align*}
            \psi(x(t_1)) & = W(t_1,x(t_1)) = W(t,x(t)) + \int_t^{t_1}\frac{d}{ds}W(s,x(s)) \,ds \\
            & = W(t,x(t)) + \int_t^{t_1} \frac{\partial}{\partial t}W(s,x(s)) + \dot{x}(s)\cdot D_xW(s,x(s)) \,ds \\
            & = W(t,x(t)) + \int_t^{t_1} \frac{\partial}{\partial t}W(s,x(s)) + f(s,x(s),u(s))\cdot D_xW(s,x(s)) \,ds \\
            & \overset{\circledast}{\geq} W(t,x(t)) - \int_t^{t_1} L(s,x(s),u(s)) \,ds  
        \end{align*}

        Then:

        \[W(t,x(t)) \leq J(t,x;u)\]

        And therefore by taking the infimum over $\mathcal{U}$ and recalling $x(t)=x$ we get:

        \[W(t,x) \leq V(t,x)\]

        If furthermore $u^{\ast}$ satisfies \ref{1-2-uoptim} then the inequality $\overset{\circledast}{\geq}$ is an equality, and therefore:

        \[W(t,x) = J(t,x;u^{\ast})\]

        Which implies that $u^{\ast}$ is optimal and $W(t,x)=J(t,x;u^{\ast})=V(t,x)$. Conversely, if $u^{\ast}$ is optimal:

        \begin{align*}
            - \int_t^{t_1} L(s,x(s),u^{\ast}(s)) \,ds & \leq \int_t^{t_1} \frac{\partial}{\partial t}W(s,x(s)) + f(s,x(s),u^{\ast}(s))\cdot D_xW(s,x(s)) \,ds \\
            & = \int_t^{t_1} \frac{\partial}{\partial t}W(s,x(s)) + f(s,x(s),u(s))\cdot D_xW(s,x(s)) \,ds = 
        \end{align*}
    \end{proof} 
\end{theorem}


Theorem \ref{1-2-Verificationthe} is an important tool in determining the explicit form of and optimal control. 
    Indeed, condition \ref{1-2-uoptim} can be restated as:

    \begin{equation}\label{1-2-optimalityconditionu}
        u^{\ast}(s) \in \argmin_{v\in U} \left\{ f(s,x^{\ast}(s),v) \cdot D_xW(s,x^{\ast}(s)) + L(s,x^{\ast}(s),v)\right\}
    \end{equation}

    For almost all $s\in[t,t_1]$.

\section{Pontryagin's principle}

We will use the notion of differentiability of the value function. 
The classical notion of differentiability which we use is the following.

\begin{definition}
    V is differentiable in $(t,x)$ if there exists $V_t(t,x),V_x(t,x)\in\R$ such that:

    \begin{equation}
        \lim_{(h,k)\to(0,0)} \frac{1}{\abs{h}+\abs{k}}\abs{V(t+h,x+k)-V(t,x)-V_t(t,x)\cdot h-V_x(t,x)\cdot k}=0
    \end{equation}
\end{definition}

Differentiability is somewhat a strong hypothesis, but it allows us to prove the folllwing
proposition.

\begin{theorem}
    Let $V$ be differentiable in $(t,x)\in Q$ and $u^{\ast}$ an optimal control
    such that $u^{\ast}\xrightarrow{s\to t}v$, then:

    \begin{equation}\label{1-3-dynamicprogrammeq}
        V_t(t,x) + L(t,x,v) + f(t,x,v)\cdot D_xV(t,x) = 0
    \end{equation}

    \begin{proof}
        Let $h>0$ s.t. $t+h<\tau$, then by \ref{1-2-propriformulation} we have:

        \[ V(t,x) = \int_t^{t+h} L(s,x(s),u^{\ast}(s)) \,ds + V(t+h,x(t+h))\]

        But because of differentiability we have:

        \[\lim_{h\to0}\frac{1}{\abs{h}}\abs{V(t+h,x(t+h))-V(t,x(t))} = V_t(t,x) + f(t,x,v)\cdot D_xV(t,x)\]

        Then we get:

        \begin{align}
        L(t,x,v) & = \lim_{h\to0} \frac{1}{\abs{h}} \int_t^{t+h} L(s,x(s),u(s)) \,ds = \lim_{h\to0} \frac{1}{\abs{h}}\abs{V(t+h,x(t+h))-V(t,x(t))} \\
        & = V_t(t,x) + f(t,x,v)\cdot D_xV(t,x)
        \end{align}
    \end{proof}
\end{theorem}

\section{Existence}

We now prove an existence theorem for optimal controls.
We study the fixed time interval case with $O=\R^n$ and the function $f$ linear in $v$.
Furthermore, we impose convexity of $L$ in $v$. Under these assumptions a classical variational
argument gives us ...
\begin{theorem}
    
\end{theorem}

\section{Infinite horizon}

A particularly interesting version of the maximization problem arising from optimal 
control is with infinite horizon. Let us consider the usual Cauchy' problem:

\begin{equation}
    \begin{cases}
        \dot{x}(s) = f(s,x(s),u(s)) & s\in[t,t_1] \\
        x(t) = x
    \end{cases}
\end{equation}

If we set $t_1=+\infty$ we get an infinite horizon problem. Thus, the maximization 
problem becomes:

\begin{equation}
    \inf_{u\in\mathcal{U}} \int_t^{\tau} L(s,x(s),u(s)) \, ds + g(\tau,x(\tau))\chi_{\tau<+\infty}
\end{equation}

Where $\tau$ is the exit time of $x(\cdot)$ from $\overline{O}$.\footnote{More precisely, is the exit time of $(s,x(s))$ from $\overline{Q}$.}

\section{Proof of Pontryagin's principle}

We will show the maximum Pontryagin's principle in the simple context of no running cost. 
A cleaver reconstruction of the non-zero running cost problem as a zero running cost one 
will enlarge the thesis to this situation. The first concept we will need is the variation 
of a control.

\begin{definition}
    Given $u\in\mathcal{U}$. For $\epsilon,r>0$ such that $0<r-\epsilon<r$ and $a\in U$ 
    we define the \textit{simple variation} $u_{\epsilon}\in\mathcal{U}$ such that:

    \begin{equation}\label{1-proofpontry-defvar}
        u_{\epsilon}(t) = \begin{cases}
            a & s\in(r-\epsilon,r) \\
            u(s) & s\notin(r-\epsilon,r)
        \end{cases}
    \end{equation}
\end{definition}

By defining the matrix $A:[0,+\infty)\rightarrow\R^{n\times n}: s\mapsto D_xf(s,x(s),u(s))$ we state 
the following lemma.

\begin{lemma}
    Let $x_{\epsilon}$ be solution of:

    \begin{equation}\label{1-proofpontry-dynamprob}
        \begin{cases}
            \dot{x}_{\epsilon}(s) = f(s,x_{\epsilon}(s),u_{\epsilon}(s)) & s\in[t,t_1] \\
            x_{\epsilon}(t) = x
        \end{cases}
    \end{equation}

    Then the solution is:

    \begin{equation}\label{1-proofpontry-solvar}
        x_{\epsilon}(s) = x(s) + \epsilon y(s) + o(\epsilon)\, \epsilon\to0
    \end{equation}

    Where $y\equiv0$ on $[t,r]$ and:

    \begin{equation}
        \begin{cases}
            \dot{y}(s) = A(s)y(s) & s\in[r,t_1]\\
            y(r) = y^r
        \end{cases}
    \end{equation}

    With $y^r=f(x(r),a)-f(x(r),u(r))$.

    \begin{proof}
        Let us divide the proof in three cases.

        \begin{itemize}
            \item $s\in[t,r-\epsilon]$: then $y(t)=0$ and $u_{\epsilon}(t)=u(t)$, therefore:
            
            \[x_{\epsilon}(t)=x(t)=x(t)+\epsilon y(t)+o(\epsilon)\]

            \item $s\in(r-\epsilon,r)$: then we have:
            
            \[x_{\epsilon}(s)-x(s) = \int_{r-\epsilon}^s f(w,x_{\epsilon}(w),u_{\epsilon}(w))-f(w,x(w),u(w)) \,dw + o(\epsilon)\]

            Which is a little \textit{o} of $\epsilon$ (because $f$ is continuous). 

            \item $s\in[r,t_1]$: from before if $s=r$ then:
            
            \begin{align}
                x_{\epsilon}(r)-x(r) & = \int_{r-\epsilon}^r f(w,x_{\epsilon}(w),u_{\epsilon}(w))-f(w,x(w),u(w)) \,dw + o(\epsilon) \\
                & = \lim_{w\to s}[f(w,x_{\epsilon}(w),u_{\epsilon}(w))-f(w,x(w),u(w))]\epsilon+ o(\epsilon) \\
                & = y^s\epsilon + o(\epsilon)
            \end{align}

            \color{red}{If $s>r$ then:}


        \end{itemize}
    \end{proof}
\end{lemma}

Let us now prove Pontryagin's principle with no running cost. The payoff functional is:

\begin{equation}\label{1-proofpontry-norunfunct}
    J(t,x;u) = \psi(x(t_1))
\end{equation}

And therefore the Hamiltonian is:

\begin{equation}\label{1-proofpontry-norunham}
    H(s,x,u,p) = - f(s,x,u)\cdot p 
\end{equation}

\begin{theorem}\label{1-proofpontry-theo}
    There exists a function $p^{\ast}:[t,t_1]\rightarrow \R^n$ such that:

    \begin{equation}
        \dot{p}^{\ast}(s) = -D_x H(s,x^{\ast}(s),u^{\ast}(s),p^{\ast}(s))\, s\in[t,t_1]
    \end{equation}

    together with the maximization:

    \begin{equation}
        H(s,x^{\ast}(s),u^{\ast}(s),p^{\ast}(s)) = \sup_{v\in U} H(s,x^{\ast}(s),v,p^{\ast}(s))
    \end{equation}

    and the trasversality condition:

    \begin{equation}
        p^{\ast}(t_1) = D\psi(x(t_1))
    \end{equation}

    \begin{proof}
        Let us drop all the $^{\ast}$. Let $p$ be the unique solution of:

        \begin{equation}
            \begin{cases}
                \dot{p}(s) = - A'(s)\cdot p(s) & s\in[t,t_1] \\
                p(t_1) = D\psi(x(t_1))
            \end{cases}
        \end{equation}

        It exists and is unique because the latter is a linear differential equation with integrable coefficient. 
        We already satisfy the trasversality condition and the adjoint dynamics. We prove the 
        maximization principle. Let $a\in U$. We define the variation $u_{\epsilon}$ for $\epsilon,r\in(t,t_1)$ as before.
        Since $\epsilon\mapsto  J(t,x;u_{\epsilon})$ for $\epsilon\in[0,1]$ has a maximum in $\epsilon=0$ we have:
        
        \begin{equation}\label{1-proofpontry-dervar0}
            \frac{d}{d\epsilon}J(t,x;u_{\epsilon})\leq0
        \end{equation}

        Computing the derivative, using \ref{1-proofpontry-defvar}:

        \begin{align}
            \frac{d}{d\epsilon}J(t,x;u_{\epsilon})\big|_{\epsilon=0} & = \frac{d}{d\epsilon}\psi(x_{\epsilon}(t_1))\big|_{\epsilon=0} \\
            & = \frac{d}{d\epsilon}\psi(x(t_1) + \epsilon y(t_1) + o(\epsilon)) = D\psi(x(t_1))\cdot y(t_1) \\
            & = p(t_1)\cdot y(t_1) = p(r)\cdot[f(r,x(r),a)-f(r,x(r),u(r))]
        \end{align}

        Where the last equality comes from:

        \begin{align*}
            \frac{d}{ds}\left(p(s)\cdot y(s)\right) & = \dot{p}(s)\cdot y(s) + p(s)\cdot \dot{y}(s) \\
            & = -A'(s)\cdot p(s)\cdot y(s) + p(s) \cdot A(s)\cdot y(s) = 0
        \end{align*}

        Therefore, by plugging into \ref{1-proofpontrydervar0} we get:

        \[0\geq p(r)\cdot[f(r,x(r),a)-f(r,x(r),u(r))]\]

        Which implies:

        \[H(r,x(r),a,p(r)) = f(r,x(r),a) \cdot p(r) \leq f(r,x(r),u(r)) \cdot p(r) = H(r,x(r),u(r),p(r))\]
    \end{proof}
\end{theorem}

Given Pontryagin's principle for no running cost problems we can extend the result to the general case:

\begin{equation}
    J(t,x;u) = \int_t^{t_1} L(s,x(s),u(s)) \,ds + \psi(x(t_1))
\end{equation}

Where the Hamiltonian is:

\begin{equation}
    H(s,x,u,p) = f(s,x,u)\cdot p + L(s,x,u)
\end{equation}

Indeed, theorem \ref{1-proofpontry-theo} holds also under these conditions. We rewrite the problem as it has no 
running cost and then apply the theorem. Let us define $x^{n+1}$ as:

% \begin{equation}
%     \begin{cases}
%         \dot{x}^{n+1}(s) = L(s,x^{n+1}(s),u(s)) & s\in [t,t_1] \\
%         x^{n+1}(0) = 0 
%     \end{cases}
% \end{equation}

\begin{equation}
    x^{n+1}(s) = \int_t^s L(w,x(w),u(w)) \,dw
\end{equation}

Then by defining $\overline{f}, \overline{g}, \overline{x},\overline{x}(s)$ as:

\begin{equation}
    \overline{f}(s,x,u)=\begin{bmatrix}
        f(s,x,u) \\
        L(s,x,u)
    \end{bmatrix},\,\overline{x}=\begin{bmatrix}
        x \\
        0
    \end{bmatrix},\,\overline{x}(s)=\begin{bmatrix}
        x(s) \\
        x^{n+1}(s)
    \end{bmatrix},\,\overline{g}(\overline{x}(t_1)) = g(x(t_1)) + x^{n+1}(t_1)
\end{equation}

Thus, the problem has no running cost. We can apply the theorem and, noticing that 
$p^{n+1}\equiv1$ we get the thesis.

\newpage

\chapter{Stochastic Optimal Control}

We now study stochastic optimal control. The system we aim to control is governed by stochastic differential equations. 

Short description of what is done in this chapter.

\section{Markovian diffusion processes}

I now recall some definitions, give new ones and set the notation. Let $\Sigma\subseteq\R^n$ and $\mathcal{B}(\Sigma)$ 
the associated Borel $\sigma$-algebra. Let $(\Omega, \mathcal{F}, P)$ a general probability space. 
Given $x(s,\omega)$ a $\Sigma$-valued random process from $I_0=[t_0,t_1)$ and $(\Omega,\mathcal{F})$, let us denote by

\[P(C\,|\,x(s_1),\dots,x(s_m)),\]

the conditional probability of $C\in\mathcal{F}$ given the sigma algebra $\bigvee_{i=1}^m\sigma(x(s_i))$.

\begin{definition}\label{2-1-markovprocessdef}
    A stochastic process $x(\cdot, \cdot)$ satisfies the Markov property if there exists a function 
    $p:I_0\times \Sigma\times I_0\times  \mathcal{B}(\Sigma)\rightarrow \R$ such that:

    \begin{enumerate}
        \item for all $t,s,B$ the function $x\mapsto p(t,x,s,B)$ is borel measurable on $\Sigma$,
        \item for all $t,x,s$ the function $A\mapsto p(t,x,s,B)$ is a probability measure on $(\Omega,\mathcal{F})$
        \item the Chapman-Kolmogorov equation holds for all $s,t,r\in I_0$, that is if $t<r<s$ then
        \begin{equation}\label{2-1-markovprocessdef-chapkol}
            p(t,x,s,B) = \int_{\Sigma} p(r,y,s,B)\,p(t,x,r,dy).
        \end{equation}
    \end{enumerate}

    Furthermore, for all $r,s\in I_0$ where $r,s$ and for all $B\in\mathcal{B}(\Sigma)$ then

    \begin{equation}\label{2-1-markovprocessdef-condonp}
        P(x(s)\in B\,|\,\mathcal{F}_r^x) = p(r,x(r),s,B),
    \end{equation}

    where $\mathcal{F}_r^x=\sigma\left(x(l)\,:\,l\in[t_0,r]\right)$.
\end{definition}

The function $p$ is called \textit{Markov Transition Kernel}. We shall see a Markov transition kernel as 
the probability that the system starting from $x$ at time $t$ will be in $B$ at time $s$. This heuristic interpretation 
clarifies the following notation

\begin{equation}
    E_{tx}\phi(x(s)) = \int_{\Sigma} \phi(y)\,p(t,x,s,dy), 
    % p(t,x,s,B) = P(x(s)\in B\,|\,)
\end{equation}

for some real valued borel-measurable function $\phi$. 

Given a Markov process $x(\cdot, \cdot)$ we can define a family of linear operators associated to it. Let $t<s$, hereafter all time indices will always be in $I_0$, and define

\begin{equation}
    T_{ts}\phi(x) = \int_{\Sigma} \phi(y)\,p(t,x,s,dy) = E_{tx}\phi(x(s)).
\end{equation}

Integrability assumptions on $\phi$ vary from case to case. For now, we can take $\phi$ to be bounded. 
Because of Chapman-Kolmogorov equation \eqref{2-1-markovprocessdef-chapkol} the family $(T_{ts})_{t,s\in I_0}$ 
satisfies the property

\begin{equation}\label{2-1-propT}
    T_{tr}\left[T_{rs}\phi\right] = T_{ts}\phi
\end{equation}

for all $t<r<s$. This family of linear operators defines another operator, the \textit{backward evolution operator}.
Let $A:\left\{\Phi:I_0\times\Sigma\rightarrow\R\right\}\rightarrow \R$ such that

\begin{equation}\label{2-1-backwarddef}
    A\Phi(t,x) = \lim_{h\to0+} \frac{E_{tx}\Phi(t+h,x(t+h))-\Phi(t,x)}{h},
\end{equation}

provided that the limit exists. We define $\mathcal{D}(A)$ as the space of functions such that limit \eqref{2-1-backwarddef} exists. 

\begin{proposition}
    Let $A$ as before, then for all $\Phi\in\mathcal{D}(A)$ the following hold:

    \begin{enumerate}
        \item $\Phi,\frac{\partial\Phi}{\partial t}$ and $A\Phi$ are continuous.
        \item For all $t,s\in\overline{I}_0$, $t<s$ then
        
        \[E_{tx}\abs{\Phi(s,x(s))}< +\infty,E_{tx}\int_t^s\abs{A\Phi(r,x(r))}\,dr < +\infty.\]
        
        \item Dynkin's formula holds for all $t<s$ that is:
        
        \begin{equation}\label{2-1-dynkform}
            E_{tx}\Phi(s,x(s)) - \Phi(t,x) = E_{tx}\int_t^s A\Phi(r,x(r)) \,dr,\,\forall t,s\in I_0,\, t<s.
        \end{equation}
    \end{enumerate}

    % \begin{proof}
    %     I prove Dynkin's formula in the case of $T_{ts}$ being a Feller semigroup. 
    %     %Dynkin, E. B., Markov processes,  

    % \end{proof}
\end{proposition}

Dynkin's formula can be proved in different instances, subject to the nature of the random process. We will see 
that it is a natural consequence of It\^o's formula for continuous state space processes.

If the random process $x$ is autonomous (time-homogeneous), then the linear operator family is a 
semigroup. Recall that a Markov process is homogeneous if for all $t<s$ in $I_0$ then

\[p(t,x,s,B) = p(0,x,s-t,B).\]

If so, by denoting $T_s=T_{0s}$, property \eqref{2-1-propT} is given by

\begin{align*}
    T_{s+r}\phi(x) & = \int_{\Sigma} \phi(y)\,p(0,x,s+r,dy) \\
    % & = \int_{\Sigma} \phi(y)\,p(r,x,s,dy) \\
    & = \int_{\Sigma}\phi(y)\int_{\Sigma} p(r,z,r+s,dy)\,p(0,x,r,dz) \\
    & = \int_{\Sigma}\int_{\Sigma} \phi(y) p(r,z,r+s,dy)\,p(0,x,r,dz) \\
    & = \int_{\Sigma}\int_{\Sigma} \phi(y) p(0,z,s,dy)\,p(0,x,r,dz) \\
    & = \int_{\Sigma} T_{s}\phi(z) \,p(0,x,r,dz) \\
    & = T_r\left[T_s\phi(x)\right].
\end{align*}

While the backward evolution operator analogous is called the \textit{generator} and is defined as

\begin{equation}
    G\phi(x) = - \lim_{h\to0^+} \frac{T_h\phi(x) - \phi(x)}{h}
\end{equation}

with $D(G)$ being as $\mathcal{D}(A)$ before. It is worth noting that, formally, the following equality holds:

\begin{equation}
    A\Phi = \frac{\partial \Phi}{\partial t} - G\Phi(t,\cdot).
\end{equation}

This relation links the two operators and the autonomous to the non-autonomous case. We now turn our attention to 
a subset of Markov processes: diffusion processes. A diffusion process is a Markov process whose paths are continuous. As Varadhan writes \cite{Varadhan} "we would like to identify them through their infinitesimal means and infinitesimal covariance", more formally. 

\begin{definition}
    A diffusion process $x:\overline{I}_0\times\Omega\rightarrow\Sigma$ is an almost surely continuous Markov process with Markov transition kernel $p$ such that:

    \begin{enumerate}
        \item For every $\epsilon>0$ then
        
        \begin{equation}
            \lim_{h\to0^+} \int_{\abs{x-y}>\epsilon} \,p(t,x,t+h,dy) = 0
        \end{equation}

        \item There exist functions $a_{ij}(t,x),f_{ij}(t,x)$ for $(t,x)\in\overline{Q}_0$ and $i,j=1,\dots,n$ such that for every $\epsilon>0$ then
        
        \begin{equation}
            \lim_{h\to0^+} \int_{\abs{x-y}\leq\epsilon} (y_i-x_i)\,p(t,x,t+h,dy) = f_i(t,x),
        \end{equation}

        and

        \begin{equation}
            \lim_{h\to0^+} \int_{\abs{x-y}\leq\epsilon} (y_i-x_i)(y_j-x_j)\,p(t,x,t+h,dy) = a_{ij}(t,x).
        \end{equation}
    \end{enumerate}

    These limits are intended uniformly.
\end{definition}

Functions $f=(f_1,\dots,f_n)$ and $a=(a_{ij})_{ij}$ are respectively called local drift and local covariance coefficients.

How does the backward evolution operator, and the generator in the autonomous case, adapt to this situation? 
To answer this question we reduce our problem to a stochastic differential one by relying on the differential structure of a diffusion process. 
Given the local drift and covariance $f,a$ of a diffusion process $x$ we claim that it satisfies

\begin{equation}\label{2-1-SDEdef}
    dx(s) = f(s,x(s))ds + \sqrt{a}(s,x(s))dw(s) 
\end{equation}

Clearly, we have to impose further conditions of the stochastic differential equation's coefficients to ensure the existence of a solution. 
In particular, we want those coefficients to be Lipshitz and sub-linearly growing with respect to the second variable. In equation \eqref{2-1-SDEdef} we define the square root of $a$ as a function $\sqrt{a}=\sigma$ such that $\sigma(t,x)\cdot\sigma'(t,x) = a(t,x)$. We recall that under existence hypothesis for every $\Phi\in C^{1,2}(\overline{Q}_0)$ It\^o's formula holds, that is:

\begin{equation}\label{2-1-itotox}
    d\Phi(s,x(s)) = \Phi_s(s,x(s))ds + \sum_{i=1}^n \Phi_{x^i}(s,x(s)) dx^i(s) + \frac{1}{2}\sum_{i,j=1}^n \Phi_{x^ix^j}(s,x(s)) d[x,x]^{ij}(s),
\end{equation}

where $[x,y]$ is the covariation of processes $x$ and $y$. Recall that this relation has always to be intended in integral form, that is:

\begin{equation}
    \begin{aligned}
        \Phi(s,x(s)) = \Phi(t,x) & + \int_t^s\Phi_s(r,x(r))\,dr \\
        & + \sum_{i=1}^n \int_t^s \Phi_{x^i}(r,x(r))\,dx^i(r) + \frac{1}{2}\sum_{i,j=1}^n \int_t^s\Phi_{x^ix^j}(r,x(r))\,d[x,x]^{ij}(r).
    \end{aligned}
\end{equation}

Via this relation, we can reconstruct Dynkin's formula in this setting. By defining the operator $A$ as in \eqref{2-1-dynkform} we have

\begin{align*}
    \Phi(s,x(s)) & = \Phi(t,x) + \int_t^s\Phi_s(r,x(r))\,dr \\
    & + \sum_{i=1}^n \left[\int_t^s \Phi_{x^i}(r,x(r))f_i(r,x(r))\,dr + \sum_{j=1}^n\int_t^s\Phi_{x^i}(r,x(r))\sigma_{ij}(r,x(r))\,dw^j(r)\right] \\
    & + \frac{1}{2}\sum_{i,j=1}^n \sum_{l=1}^n\int_t^s\Phi_{x^ix^j}(r,x(r))\sigma_{il}(r,x(r))\sigma_{jl}(r,x(r))\,dr \\
    & = \Phi(t,x) + \int_t^s \Phi_s(r,x(r)) + D_x\Phi\cdot f(r,x(r)) + \frac{1}{2}D^2_x\Phi\cdot a(r,x(r)) \,dr \\
    & + \int_t^s D_x\Phi\cdot\sigma(r,x(r)) \,dw(r), 
    % & = \Phi(t,x) + \int_t^s A\Phi(r,x(r))\,dr + \int_t^s D_x\Phi\cdot\sigma(r,x(r)) \,dw(r)
\end{align*}

but we see that the last (stochastic) integral can be seen as a martingale. In particular, if we take $\Phi$ to have polynomial growth of some order $m$, i.e.

\begin{equation}
    \abs{\Phi(t,x)} \leq K(1+\abs{x}^m)\,\forall(t,x)\in\overline{Q}_0,
\end{equation}

then $D_x\Phi\cdot\sigma\in\mathbb{L}^2(I_0)$, where:

\[\mathbb{L}^2(I_0) = \left\{x:I\times\Omega\rightarrow\Sigma\,|\,E\int_I\abs{x(s)}^2\,ds<\infty\right\},\]

and therefore its stochastic integral is a martingale (with respect to the canonical filtration associated to the Brownian motion $w$). 
Therefore, if we take the (conditional) expectation we get

\begin{equation}
    E_{tx}\Phi(s,x(s)) = \Phi(t,x) + E_{tx}\int_t^s \Phi_s(r,x(r)) + D_x\Phi\cdot f(r,x(r)) + \frac{1}{2}D^2_x\Phi\cdot a(r,x(r))\,dr.
\end{equation}

It is now coherent to define the operator $A:C_p^{1,2}(\overline{Q}_0)\rightarrow\R$ as:

\begin{equation}\label{2-1-newdefA}
    A\Phi(r,x(r)) = \Phi_s(r,x(r)) + D_x\Phi\cdot f(r,x(r)) + \frac{1}{2}D^2_x\Phi\cdot a(r,x(r)),
\end{equation}

where $C_p^{1,2}(I)$ is the family of functions $g$ from $I$ into $\R$ such that $g,g_s,g_{x_i},g_{x_ix_j}$ are continuous with polynomial growth.

\begin{remark}
    Be careful that the stochastic integral

    \[\int_t^s D_x\Phi\cdot\sigma(r,x(r))\,dw(r)\]

    is a martingale because $x$ satisfies

    \[E_{tx}\abs{x(r)}^m\leq C_m(1+\abs{x}^m)\,\forall r\in I_0,\]

    as it is solution of the SDE \ref{2-1-SDEdef}.\footnote{This is a standard result in SDE theory.}
\end{remark}

Consequently, the generator $G$ of the time-homogeneous case is defined by

\begin{equation}
    G\Phi(x) = -\frac{1}{2}\sum_{i,j=1}^n a_{ij}(x)\Phi_{x_ix_j}(x) - \sum_{i=1}^n f_i(x)\Phi_{x_i}(x).
\end{equation}\

\section{Markov control processes}

So far we talked about Markov processes without specifying any kind of control. A control process in any 
stochastic process $u:\Omega\rightarrow U$, where $U$ is the control space, that influences the evolution of the random process $x$. 
Formally, let $Q=I_0\times O$ and $u$ as before and define:

\begin{equation}\label{2-1-controlledSDE}
    \begin{cases}
        dx(r) = f(r,x(r),u(r))dr + \sigma(r,x(r),u(r))dw(r)\,r\in I_0 \\
        x(t) = x
    \end{cases}
\end{equation}

where $U\subset\R^m$ closed, $f,\sigma\in C(\overline{Q}_0\times U)$, $f(\cdot,\cdot,v),\sigma(\cdot,\cdot,v)$ belong to $C^1(\overline{Q}_0)$ for all $v\in U$, 
such that there exists $C>0$ such that:

\begin{align}
        \abs{f_t}+\abs{f_x} \leq C,\,\abs{\sigma_t}+\abs{\sigma_x}\leq C \\
        \abs{f(t,x,v)}\leq C(1+\abs{x}+\abs{v}) \\
        \abs{\sigma(t,x,v)}\leq C(1+\abs{x}+\abs{v})
\end{align}


We can relax the assumption by imposing Lipschitz condition on $t$ and $x$ for every fixed $v$. Furthermore,
we assume $u$ to be \textit{admissible}, that is:

\begin{equation}\label{2-1-condoncontrol}
    E\int_t^{t_1} \abs{u(s)}^m\,ds <\infty\,\forall m\in\mathbb{N}.
\end{equation}

It is implied by $U$ being compact. Under these hypotheses, equation \ref{2-1-controlledSDE} has a unique (indistinguishable) solution. 
Where does optimality play its role? We define running and terminal costs $L,\Psi$, both continuous and satisfying:

\begin{align}
    \abs{L(s,x,v)} \leq C(1+\abs{x}^k+\abs{v}^k) \\
    \abs{\Psi(s,x)} \leq C(1+\abs{x}^k)
\end{align}

for suitable $C,k>0$. We also define $\tau$ to be the exit time of $(s,x(s))$ from $Q$. We define:

\begin{equation}
    J(t,x;u) = E_{tx}\left\{\int_t^{\tau} L(s,x(s),u(s))\,ds + \Psi(\tau,x(\tau))\right\}
\end{equation}

for every initial condition $(t,x)\in Q$ and control $u$. We aim to minimize this criterion, that is:

\[\inf_{u\in\mathcal{U}}J(t,x;u).\]

This formulation is not mathematically formal enough, let us restate it. We begin by defining an infimum criterion with respect to a probability space, or more 
formally a \textit{probability system}, and then we'll take the infimum over all probability systems. 

\begin{definition}
    A reference probability system is a tuple $(\Omega,\{\mathcal{F}_s\},P,\omega)$ such that:
    
    \begin{enumerate}[label=\alph*)]
        \item $\nu=(\Omega,\mathcal{F}_{t_1},P)$ is a probability space
        \item $\{\mathcal{F}_s\}$ is a filtration on $\Omega$
        \item $w$ is an $\mathcal{F}$-adapted Brownian motion on $[t,t_1]$.
    \end{enumerate}

    We denote with $\mathcal{A}_{t\nu}$ the collection of all $\mathcal{F}$ progressively measurable (that is $\mathcal{B}([t,s])\times\mathcal{F}_s$-adapted), 
    $U$ valued processes $u$ such that condition \ref{2-1-condoncontrol} holds on $[t,t_1]$.
\end{definition}

We define:

\begin{equation}\label{2-1-systemopt}
    V_{\nu} = \inf_{u\in\mathcal{A}_{t\nu}} J(t,x;u)
\end{equation}

while we define:

\begin{equation}\label{2-1-opt}
    V_{PM} = \inf_{\nu} V_{\nu}.
\end{equation}

Equation \ref{2-1-systemopt} and \label{2-1-opt} respectively define $\nu$-\textit{optimality} and \textit{optimality} for those control that 
satisfy them. We adapt the definition of operator $A$ to this situation by defining for every element of the control space $v$ the functional:

\begin{equation}\label{2-1-defAwithv}
    A^{v}\Phi = \Phi_t + \sum_{i=1}^nf_i(t,x,v)\Phi_{x_i} + \frac{1}{2}\sum_{i,j=1}^n a_{ij}(t,x,v)\Phi_{x_ix_j}, \Phi\in C^{1,2}_p(\overline{Q}_0)
\end{equation}

where $a=\sigma\sigma'$. As we did in the determinist case, we provide a heuristic derivation of the Hamilton-Jacobi-Bellman equation (the verification theorem), 
and then we'll formally prove it. Let us suppose that $O=\R^n$, then $J$ is:

\begin{equation}
    J(t,x;u) = \int_t^{t_1} L(s,x(s),u(s))\,ds + \Phi(t_1,x(t_1)).
\end{equation}

By the dynamic programming principle for every $h<t_1-t$:

\[V(t,x) = \inf_{u\in\mathcal{A}}E_{tx}\left\{\int_t^{t+h} L(s,x(s),u(s))\,ds + V(t+h,x(t+h))\right\}.\]

If we take the constant control $u\equiv v$ then by Dynkin's formula we get:

\begin{align}
    0 & \leq E_{tx}V(t+h,x(t+h))-V(t,x) + E_{tx}\int_t^{t+h} L(s,x(s),v)\,ds \\
    & = E_{tx}\int_t^{t+h} A^vV(s,x(s))\,ds + E_{tx}\int_t^{t+h} L(s,x(s),v)\,ds
\end{align}

dividing by $h$ and taking the limit for $h\to0^+$:

\[0 \leq A^vV(t,x) + L(t,x,v).\]

If we take $u^{\ast}$ to be optimal, then equality holds:

\[A^{u^{\ast}}V(t,x) + L(t,x,u^{\ast}(t)) = 0.\]

We now present the verification theorem rigorously. Let us define the Hamiltonian for this problem. For every $(t,x)\in\overline{Q}_0$, 
$p\in\R^n$ and $A\in\mathcal{S}_+^n$ (set of symmetric, non-negative definite $n\times n$ matrices) we define:

\begin{equation}\label{2-1-defhamilt}
    \mathcal{H}(t,x,p,A) =  \sup_{v\in U}\left\{-f(t,x,v)\cdot p - \frac{1}{2}tr\left[a(t,x,v)\cdot A\right] - L(t,x,v)\right\}
\end{equation}

where for matrices $A,B\in\R^{n\times n}$:

\begin{equation}
    tr\left(AB\right) = \sum_{i,j=1}^n A_{ij}B_{ji}, 
\end{equation}

which is equal to $\sum_{i,j=1}^n A_{ij}B_{ij}$ for symmetric matrices.

We can now state the verification theorem using the Hamiltonian defined in \ref{2-1-defhamilt}.

\begin{theorem}
    Let $W\in C^{1,2}(Q)\cap C_p(\overline{Q})$ such that:

    \begin{align}\label{2-1-hamilcond}
        -\frac{\partial W}{\partial t}(t,x) + \mathcal{H}(t,x,D_xW,D_x^2W) = 0, & \forall(t,x)\in Q \\
        W(t,x) = \Phi(t,x), & \forall(t,x)\in\partial Q.
    \end{align}

    Then:

    \begin{enumerate}
        \item for any system $\nu$, initial condition $(t,x)\in Q$ and any $u\in \mathcal{A}_{t\nu}$ then:
        
        \begin{equation}
            W(t,x) \leq J(t,x;u)
        \end{equation}

        \item If there exists $\nu^{\ast}=(\Omega^{\ast},\{\mathcal{F}_s^{\ast}\},P^{\ast},w^{\ast})$ and $u^{\ast}\in\mathcal{A}_{t\nu^{\ast}}$ such that:
        
        \begin{equation}
            u^{\ast}(s) \in \arg\min_{v\in U}\left\{f(s,x^{\ast}(s),v)\cdot D_xW(s,x^{\ast}(s)) + \frac{1}{2}tr\left[a(s,x^{\ast}(s),v)\cdot D_x^2(s,x^{\ast}(s))\right] + L(s,x^{\ast}(s),v)\right\}
        \end{equation}

        for almost all $(s,\omega)\in[t,\tau^{\ast}]\times\Omega^{\ast}$, then:

        \begin{equation}
            V_{PM}(t,x) = J(t,x;u^{\ast}).
        \end{equation}
    \end{enumerate}

    \begin{proof}
        We assume $O$ to bounded and $W\in C^{1,2}(\overline{Q})$. Because of \ref{2-1-hamilcond} for all $s\in[t,\tau]$:

        \begin{equation}\label{2-1-proofver-keyeq}
            0\leq A^{u(s)}W(s,x(s)) + L(s,x(s),u(s)).
        \end{equation}

        Because of Ito:

        \begin{equation}
            W(\tau,x(\tau)) - W(t,x) = \int_t^{\tau} A^{u(s)}W(s,x(s))\,ds + \int_t^{\tau} D_x\Phi(s,x(s))\cdot \sigma(s,x(s),u(s))\,dw(s).
        \end{equation}

        Because of estimates on SDE solution the last stochastic integral is a $\mathcal{F}_s$-martingale. Then if we take the expectation $E_{tx}$ we get:

        \begin{align}
            0 & \leq E_{tx} \int_t^{\tau}A^{u(s)}W(s,x(s))\,ds + E_{tx}\int_t^{\tau} L(s,x(s),u(s))\,ds \\
            & = E_{tx}\left(W(\tau,x(\tau)) - W(t,x)\right) - E_{tx}\int_t^{\tau} D_x\Phi(s,x(s))\cdot \sigma(s,x(s),u(s))\,dw(s) \\
            & + E_{tx}\int_t^{\tau} L(s,x(s),u(s))\,ds \\
            & = -W(t,x) + E_{tx}\left\{\int_t^{\tau} L(s,x(s),u(s))\,ds + W(\tau,x(\tau))\right\} \\
            & = -W(t,x) + J(t,x;u).
        \end{align}

        If $O$ is unbounded we define for every $\rho>0$ such that $\rho^{-1}<t_1-t_0$ the set:

        \begin{equation}
            O_{\rho} = O\cap\left\{\abs{x}<\rho\,|\,d(x,\partial O)>\frac{1}{\rho}\right\},\, Q_{\rho} = [t_0,t_1-\rho^{-1}]\times O_{\rho}
        \end{equation}

        and $\tau_{\rho}$ the exit time from $Q_{\rho}$. Then $Q_{\rho}$ is bounded, and $W\in C^{1,2}(\overline{Q}_{\rho})$, then:

        \begin{equation}
            W(t,x) \leq E_{tx}\left\{\int_t^{\tau_{\rho}} L(s,x(s),u(s))\,ds + W(\tau_{\rho},x(\tau_{\rho}))\right\}.
        \end{equation}

        We now take $\rho\to+\infty$ and get the thesis. We have convergence in probability for $\tau_{\rho}\xrightarrow{\rho\to+\infty}\tau$. 
        We prove uniform integrability of the rhs and therefore get $L^1$ convergence. We have:
        
        \begin{align}
            E_{tx}\int_t^{\tau_{\rho}} \abs{L(s,x(s),u(s))}\,ds & \leq E_{tx}\int_t^{t_1}\abs{L(s,x(s),u(s))}\,ds \\
            & \leq E_{tx}\int_t^{t_1}\left(1+\abs{x(s)}^k+\abs{u(s)}^k\right)\,ds < +\infty    
        \end{align}
        
        because $u$ is admissible and estimates on SDE solutions. While we have:

        \begin{align}
            E_{tx}\abs{W(\tau_{\rho},x(\tau_{\rho}))}^{\alpha} & \leq K E_{tx}\left(1+\abs{x(\tau_{\rho})}^k\right)^{\alpha} \\
            & \leq 2^{\alpha-1}K\left(1+E_{tx}\norm{x}^{\alpha k}\right) \leq C
        \end{align}

        for $\alpha>\frac{1}{k}$ and estimates on SDE solutions. Therefore we get:

        \begin{equation}
            \lim_{\rho\to+\infty} E_{tx}\left\{\int_t^{\tau_{\rho}} L(s,x(s),u(s))\,ds + W(\tau_{\rho},x(\tau_{\rho}))\right\} = E_{tx}\left\{\int_t^{\tau} L(s,x(s),u(s))\,ds + W(\tau,x(\tau))\right\}.
        \end{equation}

        Part $b)$ comes from equality in equation \ref{2-1-proofver-keyeq}.
    \end{proof}
\end{theorem}

% THEN DYNAMIC PROGRAMMING APPROACH AND VERIFICATION THEOREM: i present the formal derivation via $A$. Then the theorem. 
% THEN CONTROLLED MARKOV PROCESSES.

% I HAVE TO SEARCH FOR A MORE GENERAL VERSION OF EXISTENCE THEOREM for markov control policies.


% THEN EXAMPLE IF TIME ALLOWS.

% THEN REREAD, REPEAT. I DONT THINK I'LL DO SLIDES. MAYBE THEY CAN BE A GOOD ASSET. LESS INFO, LESS THINGS TO KNOW PERFECTLY. BUT I HAVE TO KNOW EVERYTHING!!!

\section{Stochastic Maximum Principle}

The stochastic analogous of Pontryagin's principle is the stochastic maximum principle. It relies on the notion of 
backward stochastic differential equation, whose solution will provide a necessary condition on the controlled system. 

\subsection{Backward Stochastic Differential Equation}

A backward stochastic differential equation is a SDE where the initial date is replaced by a final distribution. We 
start by defining a formal concept of solution and provide a general result about existence and uniqueness. 

We work with a filtered probability space $(\Omega,\mathcal{F}, P, \left\{\mathcal{F}_s\right\})$ and a Brownian motion $(w_s)$ adapted 
to the space filtration. We assume that the filtration is the natural one associated to $w$. A BSDE has the form:

\begin{equation}\label{2-1-defBSDE}
    \begin{cases}
        - dy_s = f(s,y_s,z_s)ds - z_sdw_s & \forall s\in[t,t_1]\\
        y_{t_1} = \xi, 
    \end{cases}
\end{equation}

where $f$ is real valued and $\xi$ a suitable random variable. Further conditions on $f$ and $\xi$ will be imposed 
by the existence theorem. The above definitions has to be intended in integral form. To do so we have to specify some integrability conditions. Let us define the space:

\begin{equation}\label{2-1-defS}
    \mathbb{S}^2(t,t_1) = \left\{(X_s)_{t\in[t,t_1]}\,|\, X_s\in\R \text{ is progressively measurable},\, E\left[\sup_{s\in[t,t_1]}\abs{X_s}^2\right]<+\infty\right\}
\end{equation}

and:

\begin{equation}\label{2-1-defH}
    \mathbb{H}^2(t,t_1)^n = \left\{(X_s)_{t\in[t,t_1]}\,|\, X_s\in\R^n \text{ is progressively measurable},\, E\left[\int_t^{t_1} \abs{Y_s}^2\,ds\right]<+\infty\right\}.    
\end{equation}

We can now define the solution concept of \ref{2-1-defBSDE}.

\begin{definition}
    A solution of \ref{2-1-solBSDE} is a couple $(y,z)\in\mathbb{S}^2(t,t_1)\times\mathbb{H}^2(t,t_1)^n$ such that:
    
    \[y_s = \xi + \int_s^{t_1} f(r,y_r,z_r)\,dr - \int_s^{t_1} z_r\,dw_r\] 

    holds for all $s\in[t,t_1]$. 
\end{definition}

Some measurability and integrability conditions are necessary for equation \ref{2-1-defBSDE} to make sense. Existence 
of a solution will be obtained through a classical fixed point method, which will rise from Lipschitz condition on $f$. 
We denote by $m$ the Lebesgue measure on $[t,t_1]$.

\begin{theorem}
    Let $f:\Omega\times[t,t_1]\times\R\times\R^n\rightarrow \R$ such that $f(\cdot,\cdot,y,z)$ is progressively measurable for all 
    $(y,z)\in\R^{n+1}$, $f(\cdot,\cdot,0,0)\in\mathbb{H}^2(t,t_1)^1$ and there exists $C>0$ such that:

    \begin{equation}
        \abs{f(s,y_1,z_1)-f(s,y_2,z_2)} \leq C(\abs{y_1-y_2} + \abs{z_1-z_2}),\,\forall y_1,y_2,z_1,z_2,\, m\otimes P\,a.s.
    \end{equation}

    Then for every $\xi\in L^2$ the BSDE \ref{2-1-defBSDE} has a unique solution.

    \begin{proof}
        We use completeness of the space $(\mathbb{S}(t,t_1)\times\mathbb{H}^2(t,t_1)^n, \norm{\cdot}_{\beta})$ where:

        \begin{equation}
            \norm{(y,z)}_{\beta} = \left(E\left[\int_t^{t_1} e^{\beta s}\left(\abs{y_s}^2+\abs{z_s}^2\right)ds\right]\right).
        \end{equation}

        This result is proved in the appendix. We call the previous Banach space $(X,\norm{\cdot})$. We construct the map 
        $\Phi:X\rightarrow X$ defined as $\Phi(u,v)=(y,z)$. The processes $y$ and $z$ are defined as follows. We define:

        \begin{equation}
            M_s = E\left[\xi + \int_0^{t_1} f(r,u_r,v_r)\,dr\,|\,\mathcal{F}_s\right].
        \end{equation}

        It is a square integrable martingale because:
        
        \begin{align}
            E\left\{E^2\left[\xi + \int_t^{t_1} f(r,u_r,v_r)\,dr\,|\,\mathcal{F}_s\right]\right\} & 
            % \leq C E^2\left\{E\left[\xi + \int_t^{t_1} f(r,u_r,v_r)\,dr\,|\,\mathcal{F_s}\right]\right\} \\
            % & = C E^2\left[\xi + \int_t^{t_1} f(r,u_r,v_r)\,dr\right] \\
             \leq C_0 E^2\xi + C_1E^2\int_t^{t_1} f(r,0,0)\,dr \\
            & + C_2E^2\int_t^{t_1} f(r,u_r,v_r)-f(r,0,0)\,dr \\
            & + C_3E\xi^2E\int_t^{t_1}f(r,0,0)^2\,dr \\
            & + C_4E\xi^2E\int_t^{t_1}(f(r,u_r,v_r)-f(r,0,0))^2\,dr, 
        \end{align}

        which is finite because of the assumptions and given $(u,v)\in X$. It is a martingale. Therefore, by 
        Ito's martingale representation theorem there exists a unique $z\in\mathbb{H}^2(t,t_1)^n$\footnote{Unique with respect to $\norm{\cdot}_{\mathbb{H}}$.} and $m_t\in L2$ such that:
        
        \begin{equation}
            m_s = m_t + \int_t^{s} z_r\,dw_r.
        \end{equation}

        We define $(y,z)$ as $z$ being the unique process of the martingale representation of $m_s$ while $y$ to be:

        \begin{equation}
            y_s = E\left[\xi + \int_s^{t_1} f(r,u_r,v_r)\,dr\,|\,\mathcal{F}_s\right] = m_s - \int_t^s f(r,u_r,v_r)\,dr.
        \end{equation}

        We know that $z\in\mathbb{H}^2$. We show $y\in\mathbb{S}^2$:

        \begin{align}
            E\left[\sup_{s\in[t,t_1]}\abs{y_s}^2\right] & \leq C_0E\left[\sup_{s\in[t,t_1]}\abs{m_s}^2\right] + C_1E\left[\sup_{s\in[t,t_1]}\abs{\int_t^s f(r,u_r,v_r)\,dr}^2\right],
        \end{align}

        but as before the second addend converges while for the first one:

        \[E\left[\sup_{s\in[t,t_1]}\abs{m_s}^2\right] \leq C_0 Em_t^2 + C_1E^{1/2}m_t^2E^{1/2}\sup_{s\in[t,t_1]}\left[\int_t^{s}z_r\,dw_r\right]^2+C_3E^{1/2}\sup_{s\in[t,t_1]}\left[\int_t^{s}z_r\,dw_r\right]^2\]

        which converges beacuse Doob's inequality implies:

        \[E\left[\sup_{s\in[t,t_1]}\int_t^sz_r\,dw_r\right]^2 \leq 4E\left[\int_t^{t_1}z_r^2\,dw_r\right]<+\infty.\]

        If we prove $\Phi$ to be a strict contraction we'll have a unique (in $(X,\norm{\cdot})$) fixed point, therefore the thesis. 
        
        Let $(U_1,V_1),(U_2,V_2)\in X$, $(X_1,Y_1),(X_2,Y_2)$ their images and $\overline{U},\overline{V},\overline{X},\overline{Y},\overline{f}_t$ the differences between subscript $1$ and $2$. 
        By Ito's formula we have:

        \begin{align}
            \abs{\overline{y_t}^2} & = -\int_t^{t_1} \frac{d}{dr}\left(e^{\beta r}\overline{y_r}^2\right) \,dr \\
            & 
        \end{align}
    \end{proof}
\end{theorem}

A notable case is the one with a generator $f$ linear in $y$ and $z$. That is, there exists 
$a,b$ bounded progressively measurable processes valued in $\R$ and $\R^n$ and $c\in\mathbb{H}(t,t_1)^1$ which define:

\begin{equation}\label{2-1-defBSDE-linear}-dy_s = \left(a_sy_s + z_sb_s + c_s\right)ds - z_sdw_s,\,y_{t_1}=\xi.\end{equation}

\begin{proposition}
    The unique solution $(y,z)$ of \ref{2-1-defBSDE-linear} is:
    
    \begin{equation}\label{2-1-linearfsol}
        \Gamma_sy_s = E\left[\Gamma_{t_1}\xi + \int_s^{t_1} \Gamma_rc_r\,dr\,|\,\mathcal{F}_s\right],
    \end{equation}

    and $z_s$ is defined via the Martingale representation of \ref{2-1-linearfsol}. The process $\Gamma$ is defined by:

    \[d\Gamma_s = \Gamma_s\left(a_sds + b_sdw_s\right),\,\Gamma_0=1.\]
\end{proposition}

\newpage

\chapter{Viscosity Solutions}

We now study viscosity solutions and their relation with optimal control.

\input{Intro - Visco.tex}

\newpage

\nocite{*}
\bibstyle{numeric}
\printbibliography
\addcontentsline{toc}{chapter}{Bibliography}

% % Run this to get table of contents
% biber Main
% pdflatex Main

\end{document}